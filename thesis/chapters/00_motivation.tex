\section{Motivation}

\paragraph{Relevanz}
Die brownsche Bewegung ist ein zentrales Konzept in der Stochastik und findet Anwendung in den Naturwissenschaften und insbesondere in der Finanzmathematik. Eine Modifikation ist die geometrische brownsche Bewegung. 
Sie dient als Grundlage für die Modellierung von Aktienkursen und anderen Finanz-Zeitreihen. Eine Vielzahl von
aktuellen Aktienkurs-Modellen basieren auf dieser Theorie.

\paragraph{Ablauf}
Die Arbeit beginnt mit grundlegenden Definitionen zu stochastischen Prozessen (Kapitel 2), inklusive Zufallsspaziergang, Binomialmodell sowie Filtration, 
bedingtem Erwartungswert, Markov- und Martingal-Eigenschaften. Kapitel 3 konstruiert die brownsche Bewegung 
als Grenzprozess diskreter Normalverteilungssummen und diskutiert zentrale Eigenschaften wie Stetigkeit und 
Selbstähnlichkeit. Darauf aufbauend wird in Kapitel 4 die geometrische brownsche Bewegung aus dem Binomialmodell 
via Taylor-Approximation und Grenzwertsätzen hergeleitet und die Log-Normalverteilung von Kursen gezeigt. 
Kapitel 5 widmet sich Anwendungen auf Zeitreihen: Kalibrierung von $\mu$ und $\sigma$, 
Konfidenzintervalle/-bänder, Backtests und Fehlermaße sowie Monte-Carlo Simulation. In 
Kapitel 6 werden Aktienoptionen behandelt: finanzmathematische Grundlagen, Bewertung im 
(erweiterten) Binomialmodell unter dem risikoneutralen Maß, der Grenzfall zur Black–Scholes-Formel sowie 
Monte-Carlo-Bewertung im allgemeinen Fall. Kapitel 7 dient als Ausblick: Stochastische Differentialgleichungen werden formal eingeführt und
aktuelle Aktienkursmodelle damit beschrieben. Beispielhaft wird das CEV-Modell auf verschiedene Kurse kalibriert und mit der geometrischen
Brownschen Bewegung verglichen.
