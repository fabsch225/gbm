\section{Motivation}

Die brownsche Bewegung stellt ein zentrales Fundament stochastischer Modellierung dar und verbindet theoretische Mathematik mit praktischen Anwendungen.
In den Naturwissenschaften wird sie zur Beschreibung zufälliger Bewegungen von Teilchen eingesetzt, während sie in der Finanzmathematik die Grundlage für die Modellierung von Kursverläufen bildet.
Eine zentrale Erweiterung ist die geometrische brownsche Bewegung, die Aktienkurse und andere Finanz-Zeitreihen realistisch abbildet und damit den Kern moderner Bewertungsmodelle darstellt.


Die Arbeit beginnt mit grundlegenden Definitionen zu stochastischen Prozessen (Kapitel 2), inklusive Zufallsspaziergang, Binomialmodell sowie Filtration, 
bedingtem Erwartungswert, Markov- und Martingal-Eigenschaften. Kapitel 3 konstruiert die brownsche Bewegung 
als Grenzprozess diskreter Normalverteilungssummen und diskutiert zentrale Eigenschaften wie Stetigkeit und 
Selbstähnlichkeit. Darauf aufbauend wird in Kapitel 4 die geometrische brownsche Bewegung aus einer diskreten Übergangsgleichung via Taylor-Approximation und Grenzwertsätzen hergeleitet und die Log-Normalverteilung von Kursen gezeigt. 
Kapitel 5 widmet sich Anwendungen auf Zeitreihen: Kalibrierung von $\mu$ und $\sigma$, 
Konfidenzintervalle/-bänder, Backtests und Fehlermaße sowie Monte-Carlo Simulation. In 
Kapitel 6 werden Aktienoptionen behandelt: finanzmathematische Grundlagen, Bewertung im 
(erweiterten) Binomialmodell unter dem risikoneutralen Maß, der Grenzfall zur Black–Scholes-Formel sowie 
Monte-Carlo-Bewertung im allgemeinen Fall. Kapitel 7 dient als Ausblick: Stochastische Differentialgleichungen werden formal eingeführt und
aktuelle Aktienkursmodelle damit beschrieben. Beispielhaft wird das CEV-Modell auf verschiedene Kurse kalibriert und mit der geometrischen
brownschen Bewegung verglichen. Anmerkungen zur Notation und zu verwendeten stochastischen Ergebnissen sind im Anhang aufgeführt. 
