\section{Motivation}

\paragraph{Relevanz}
Die Brownsche Bewegung ist ein zentrales Konzept in der Stochastik und findet Anwendung in den Naturwissenschaften, und insbesondere in der Finanzmathematik. Eine Modifikation ist die geometrische Brownsche Bewegung. 
Sie dient als Grundlage für die Modellierung von Aktienkursen und anderen finanziellen Zeitreihen. Eine Vielzahl von
aktuellen Aktienkurs-Modellen basieren auf dieser Theorie.

\paragraph{Ablauf}
Die Arbeit beginnt mit grundlegender Notation und Begriffen (Kapitel 2) und führt anschließend in s
tochastische Prozesse ein (Kapitel 3), inklusive Zufallsspaziergang, Binomialmodell sowie Filtration, 
bedingtem Erwartungswert, Markov- und Martingaleigenschaften. Kapitel 4 konstruiert die Brownsche Bewegung 
als Grenzprozess diskreter Normalverteilungssummen und diskutiert zentrale Eigenschaften wie Stetigkeit und 
Selbstähnlichkeit. Darauf aufbauend wird in Kapitel 5 die geometrische Brownsche Bewegung aus dem Binomialmodell 
via Taylor-Approximation und Grenzwertsätzen hergeleitet und die Lognormalverteilung von Kursen gezeigt. 
Kapitel 6 widmet sich Anwendungen auf Zeitreihen: Kalibrierung von $\mu$ und $\sigma$, 
Konfidenzintervalle/-bänder, Backtests und Fehlermaße sowie Monte-Carlo-Simulation. In 
Kapitel 7 werden Aktienoptionen behandelt: finanzmathematische Grundlagen, Bewertung im 
(erweiterten) Binomialmodell unter risikoneutralem Maß, der Grenzfall zur Black–Scholes-Formel sowie 
Monte-Carlo-Bewertung (auch pfadabhängig). Kapitel 8 schließt mit Fazit, methodischer Einordnung, 
weiterführenden Aspekten und Ausblick.
