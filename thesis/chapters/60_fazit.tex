\section{Fazit}

Die Arbeit spannte den Bogen von elementaren stochastischen Prozessen über das Binomialmodell und die diskrete Brownsche 
Bewegung bis zur geometrischen Brownschen Bewegung als kontinuierlichem Grenzfall. Zentrale stochastische Begriffe wie 
Filtration, bedingter Erwartungswert und Martingal wurden eingeführt und in diskreten Wahrscheinlichkeitsräumen verankert. 
Durch Grenzwertbetrachtungen werden die Konzepte dann auf kontinuierliche Wahrscheinlichkeitsräume übertragen. Es wurde gezeigt, dass im Grenzübergang $\Delta t \to 0$ die Kursdynamik durch
$S_T = S_0 \exp\!\big((\mu - \tfrac12\sigma^2)T + \sigma W_T\big)$
beschrieben wird, sodass $\log S_T$ normal- und $S_T$ log-normalverteilt ist. 

Im erweiterten Binomialmodell wurde der diskontierte Aktienkurs als Martingal unter einem risikoneutralen Maß konstruiert und die Optionsbewertung 
via Rückwärtsinduktion hergeleitet; im Grenzfall führt dies zum Black–Scholes-Modell.

Empirisch wurden die Parameter aus 
Log-Renditen geschätzt, daraus Konfidenzintervalle und -bänder abgeleitet und mittels Monte-Carlo-Simulation validiert. 
Ein Backtest auf DAX-Daten zeigte eine hohe Überdeckungsrate im 50\%-Band und nachvollziehbare Fehlermaße (MSE, MAPE, NRMSE), 
was die Praxistauglichkeit trotz der Modellvereinfachungen unterstreicht. 

Letztlich wurde gezeigt, wie die diskrete Modellierung durch Grenzübergang zur stochastischen Differentialgleichung führt, deren Lösung die 
geometrische brownsche Bewegung ist. Die Konstruktion des Itô-Integrals wird skizziert und die 
heuristische Schreibweise $dS_t = a(S_t,t)\,dt + b(S_t,t)\,dW_t$ mathematisch präzisiert.

Darauf aufbauend werden alternative Modelle wie lokale und stochastische Volatilität (z.B. CEV- und Heston-Modell), Sprung-Diffusions- und Regimewechselmodelle vorgestellt, die realistische Markteigenschaften wie Volatilitäts-Clustering oder Sprünge abbilden können. 
Am Beispiel des CEV-Modells wird die Parameterschätzung aus diskreten Daten mittels (Quasi-)Maximum-Likelihood erläutert und in R implementiert. Ein Backtest und ein Vergleich mit der geometrischen Brownschen Bewegung zeigen, dass komplexere Modelle wie das CEV-Modell bei großen Datenmengen eine bessere Prognosegüte liefern, während die GBM bei kleineren Datensätzen robuster ist. Die Ergebnisse unterstreichen die Bedeutung der Modellauswahl und Kalibrierung für die praktische Anwendung in der Finanzmathematik.


\paragraph{Methodik}
Methodisch verbindet die Arbeit diskrete Grenzwert- und Martingalargumente mit 
reproduzierbarer Empirie in R. Zudem spielte die Monte-Carlo-Simulation eine zentrale Rolle als universelles Werkzeug zur Approximation risikoneutraler Erwartungswerte, sowohl für Endwert-Auszahlungen als auch für pfadabhängige Produkte. Für die geometrische brownsche Bewegung wurden Endwerte unter dem risikoneutralen Maß exakt gezogen; bei Pfadabhängigkeiten wurde zeitlich diskretisiert. Die beobachtete Annäherung der empirischen Quantile an die analytischen Lognormal-Quantile mit wachsender Pfadzahl bestätigt Konsistenz und Korrektheit der Implementierung; der Vergleich analytischer Konfidenzbänder mit Simulationsquantilen dient als robuster Plausibilitätscheck.
Als Beispiel dient ein Datensatz von DAX-Renditen.

Die geometrische brownsche Bewegung wird über Logarithmierung, Taylor-Entwicklung bis Ordnung zwei sowie Gesetz der großen Zahlen und 
zentralen Grenzwertsatz hergeleitet. Zudem kommen die Sätze von Cramér-Wold und Pratt zum Einsatz, 
genauso wie das "Reihenkriterium für fast sichere Konvergenz", welches 
ein Korollar zum Lemma von Borel-Cantelli ist (vgl. \cite{henze}). Insbesondere wurde das Kalkül der stochastischen Differentialgleichung zunächst vermieden.

Einen Einblick in die Thematik bietet Kapitel 8: die Methodik wird um die numerische Simulation und Parameterschätzung stochastischer Differentialgleichungen (SDEs) mittels Euler-Maruyama-Verfahren und Maximum-Likelihood-Ansätzen erweitert. 
Die Implementierung alternativer Modelle die durch SDEs beschrieben werden wurde am Beispiel des CEV-Modell in R durchgeführt.
Dabei kamen die Pakete \texttt{Sim.DiffProc} und das numerische Optimierungspaket \texttt{nloptr} zum Einsatz.
Anschließend wurde das CEV-Modell mit der GBM verglichen, inklusive Backtests mit empirischen Gütemaßen.
