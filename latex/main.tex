\documentclass{article}
\usepackage{graphicx}
\usepackage{tikz}
\usepackage[utf8]{inputenc}
\usepackage[ngerman]{babel}
\usepackage{float}
\usepackage{csquotes}
\usepackage{amsmath}
\usepackage{rotating}
\usepackage{chngcntr} % more control over counters

\numberwithin{equation}{subsection} % equations numbered as section.subsection.equation

\usepackage{amsthm}
\usepackage{amssymb}
\usepackage{lipsum} 
\usepackage{wrapfig}
\usepackage{xparse}
\usepackage{placeins}
\usepackage{enumitem}
\usepackage{tabularx}
\usepackage{ragged2e}
\usepackage{xurl}
\usepackage[hidelinks]{hyperref}
\usepackage{breakurl}
\usepackage{mathabx}
\usepackage{csvsimple,booktabs}

\usepackage[twoside]{fancyhdr}
\fancyhf{} 

\renewcommand{\headrulewidth}{0.5pt}
\renewcommand{\sectionmark}[1]{\markboth{#1}{}}
\renewcommand{\subsectionmark}[1]{\markright{#1}}

\fancyhead[LE]{\nouppercase{\leftmark}}
\fancyhead[RE]{\thepage}

\fancyhead[LO]{\thepage}
\fancyhead[RO]{\nouppercase{\rightmark}}

\usepackage[linesnumbered,ruled,vlined]{algorithm2e}
\renewcommand{\algorithmcfname}{Algorithmus}
\pagestyle{fancy}

\usepackage{caption}
\usepackage{biblatex}
\addbibresource{misc/literature.bib} 

%\usepackage{tocloft}
%\setlength{\cftbeforesecskip}{0.45em} % space before each \section
%\setlength{\cftbeforesubsecskip}{0.05em} % space before each \subsection


\title{Die geometrische Brownsche Bewegung und Anwendungen}
\author{Fabian Schuller}
\date{Oktober 2025}

\usetikzlibrary{arrows.meta, positioning}

%\usepackage{xcolor}
%\usepackage[dvipsnames]{xcolor}
%\pagecolor[rgb]{0,0,0} %black
%\color[rgb]{0.5,0.5,0.5} %grey 


\newtheoremstyle{mystyle}
  {10pt}                    % Space above
  {10pt}                    % Space below
  {\normalfont}             % Body font
  {}                        % Indent amount
  {\bfseries}               % Theorem head font
  {.}                       % Punctuation after theorem head
  {15pt}                     % Space after theorem head
  {\thmname{#1}\thmnumber{ #2}\bfseries \thmnote{ (#3)}}

\usepackage{xcolor}

\definecolor{codegray}{gray}{0.95}
\definecolor{commentgreen}{rgb}{0,0.6,0}
\definecolor{keywordblue}{rgb}{0,0,1}


\theoremstyle{mystyle}

\newtheorem{satz}{Satz}[subsection]
\newtheorem{lemma}[satz]{Lemma}
\newtheorem{korr}[satz]{Korollar}
\newtheorem{defi}[satz]{Definition}
\newtheorem{defprop}[satz]{Definition und Lemma}
\newtheorem{bem}[satz]{Bemerkung}
\newtheorem{bsp}[satz]{Beispiel}

\newcommand{\lt}{\ensuremath <}
\newcommand{\gt}{\ensuremath >}

\usepackage[utf8]{inputenc}
\usepackage[T1]{fontenc}
\usepackage{listings}

\lstdefinelanguage{R}{
  keywords={if,else,repeat,while,function,for,in,next,break,TRUE,FALSE,NULL,NA,NaN,Inf},
  otherkeywords={!,!=,~,\*,\&,\%/\%,\%*\%,\%>\%,<-,<<-,->,->>,=,<=,>=,::,:::},
  sensitive=true,
  morecomment=[l]{\#},
  morestring=[b]"
}

\lstdefinestyle{r-minimal}{
  language=R,
  basicstyle=\ttfamily\small,
  numbers=left,
  numberstyle=\tiny,
  numbersep=6pt,
  stepnumber=1,
  showstringspaces=false,
  breaklines=true,
  keepspaces=true,
  columns=fullflexible,
  frame=none,
  xleftmargin=2em,
  framexleftmargin=1.5em,
  literate={ä}{{\"a}}1 {ö}{{\"o}}1 {ü}{{\"u}}1 
           {Ä}{{\"A}}1 {Ö}{{\"O}}1 {Ü}{{\"U}}1 {ß}{{\ss}}1
}

\lstset{style=r-minimal}

\begin{document}

%\pagecolor{ForestGreen!10!white} % Sets a very light forest green background for all pages
%\color{NavyBlue}                 % Sets the default text color to Navy Blue for the entire document

\begin{center}
    \vspace*{2cm}
    {\LARGE \textbf{Die geometrische Brownsche Bewegung und Anwendungen}}\\[2cm]

    {\large Diese Bachelorarbeit wurde vorgelegt am\\[0.5cm]
    
    Fachbereich 9 \\[0.3cm]
    Medizintechnik und Technomathematik\\[0.3cm]
    FH Aachen, Campus Jülich\\[0.9cm]
   

    von\\[0.9cm]
    Fabian Schuller\\
    Matrikelnummer: 3646801\\[0.9cm]

    und wurde betreut von\\[0.9cm]

    Erstprüfer: Prof. Dr. habil. Daniel Gaigall\\[0.3cm]
    Zweitprüfer: Thorsten Adrian, MSc\\[2cm]
}
    1. Oktober 2025

    \vfill
\end{center}

\thispagestyle{empty}

\newpage

\section*{Eidesstattliche Erklärung}

\noindent
Hiermit versichere ich, dass ich die Bachelorarbeit mit dem Titel \\[0.5cm] \textbf{Die geometrische Brownsche Bewegung und Anwendungen} \\[0.5cm]selbstständig verfasst und keine anderen als die angegebenen Quellen und Hilfsmittel benutzt habe, alle Ausführungen, die anderen Schriften wörtlich oder sinngemäß entnommen wurden, kenntlich gemacht sind und die Arbeit in gleicher oder ähnlicher Fassung noch nicht Bestandteil einer Studien- oder Prüfungsleistung war. Ich verpflichte mich, ein Exemplar der Bachelorarbeit fünf Jahre aufzubewahren und auf Verlangen dem Prüfungsamt des Fachbereiches Medizintechnik und Technomathematik auszuhändigen.
\\[2cm]

\noindent

\thispagestyle{empty}

\begin{tabularx}{\textwidth}{@{}lX@{}}
Aachen, den 1. Oktober 2025 
& 
\begin{flushright}

  \rule{6cm}{0.4pt} \\[-1.5cm] % Strich
  \includegraphics[height=1.8cm]{images/unterschrift.png} \\
  Fabian Schuller
\end{flushright}

\end{tabularx}

\newpage

\tableofcontents
\newpage

\section{Motivation}

Die Brownsche Bewegung stellt ein zentrales Fundament stochastischer Modellierung dar und verbindet theoretische Mathematik mit praktischen Anwendungen.
In den Naturwissenschaften wird sie zur Beschreibung zufälliger Bewegungen von Teilchen eingesetzt, während sie in der Finanzmathematik die Grundlage für die Modellierung von Kursverläufen bildet.
Eine zentrale Erweiterung ist die geometrische Brownsche Bewegung, die Aktienkurse und andere Finanz-Zeitreihen realistisch abbildet und damit den Kern moderner Bewertungsmodelle darstellt.


Die Arbeit beginnt mit grundlegenden Definitionen zu stochastischen Prozessen (Kapitel 2), inklusive Zufallsspaziergang, Binomialmodell sowie Filtration, 
bedingtem Erwartungswert, Markov- und Martingal-Eigenschaften. Kapitel 3 konstruiert die Brownsche Bewegung 
als Grenzprozess diskreter Normalverteilungssummen und diskutiert zentrale Eigenschaften wie Stetigkeit und 
Selbstähnlichkeit. Darauf aufbauend wird in Kapitel 4 die geometrische Brownsche Bewegung aus einer diskreten Übergangsgleichung via Taylor-Approximation und Grenzwertsätzen hergeleitet und die Log-Normalverteilung von Kursen gezeigt. 
Kapitel 5 widmet sich Anwendungen auf Zeitreihen: Kalibrierung von $\mu$ und $\sigma$, 
Konfidenzintervalle/-bänder, Backtests und Fehlermaße sowie Monte-Carlo Simulation. In 
Kapitel 6 werden Aktienoptionen behandelt: finanzmathematische Grundlagen, Bewertung im 
(erweiterten) Binomialmodell unter dem risikoneutralen Maß, der Grenzfall zur Black–Scholes-Formel sowie 
Monte-Carlo-Bewertung im allgemeinen Fall. Kapitel 7 dient als Ausblick: Stochastische Differentialgleichungen werden formal eingeführt und
aktuelle Aktienkursmodelle damit beschrieben. Beispielhaft wird das CEV-Modell auf verschiedene Kurse kalibriert und mit der geometrischen
Brownschen Bewegung verglichen. Anmerkungen zur Notation und zu verwendeten stochastischen Ergebnissen sind im Anhang aufgeführt. 


\section{Notation}
\begin{itemize}
    \item $E(X)$ bzw. $E[X]$ - Erwartungswert
    \item $V(X)$ bzw. $V[X]$ - Varianz
    \item $X \sim N(\mu, \sigma^2)$ - normalverteilte Zufallsvariable mit Erwartungswert $\mu$ und Varianz $\sigma^2$
    \item $(\Omega, \mathcal F, P)$ - Wahrscheinlichkeitsraum
    \item $W_t$ - Brownsche Bewegung (Wiener-Prozess)
\end{itemize}

\paragraph{Konvergenzbegriffe}

\begin{itemize}
    \item $X_n \xrightarrow{pktw.} X$ - punktweise Konvergenz (Für jedes $\omega \in \Omega$ gilt: $X_n(\omega) \to X(\omega)$)
    \item $X_n \xrightarrow{glm.} X$ - gleichmäßige Konvergenz (Für jede $\epsilon > 0$ gilt: $\sup_{\omega \in \Omega} |X_n(\omega) - X(\omega)| < \epsilon$ für $n$ groß genug)
    \item $X_n \xrightarrow{d} X$ - Konvergenz in Verteilung (Die Verteilungsfunktion $F_{X_n}$ von $X_n$ konvergiert punktweise gegen die Verteilungsfunktion $F_X$ von $X$, mindestens an den Stetigkeitsstellen von $F_X$)
    \item $X_n \xrightarrow{p} X$ - Konvergenz in Wahrscheinlichkeit (Für jede $\epsilon > 0$ gilt: $P(|X_n - X| > \epsilon) \to 0$)
    \item $X_n \xrightarrow{f.s.} X$ - fast sichere Konvergenz (Für fast alle $\omega \in \Omega$ gilt: $X_n(\omega) \to X(\omega)$)
\end{itemize}

Auf das Verhältnis zwischen den Konvergenzbegriffen wird bei Bedarf eingegangen.

\section{Stochastische Prozesse}

In einer einführenden Vorlesung zur Stochastik betrachtet man zunächst einzelne Zufallsvariablen, 
etwa das Ergebnis eines Würfelwurfs. In dieser Arbeit stehen hingegen Folgen von 
Zufallsvariablen im Mittelpunkt. Solche Folgen erlauben es, zeitliche Entwicklungen zu modellieren,
beispielsweise Veränderungen eines Systems, die zu bestimmten Zeitpunkten $t_0, t_1, \dots$ gemessen werden.
Man spricht dann von einem zeitdiskreten stochastischen Prozess. Eine natürliche 
Verallgemeinerung bilden zeitstetige stochastische Prozesse: Hierbei definiert man eine 
Familie von Zufallsvariablen $(X_t)_{t \in \mathbb{R}_{\ge 0}}$, wobei $t$ kontinuierlich als 
Zeitparameter interpretiert wird. Im Folgenden werden zwei wichtige Beispiele für stochastische Prozesse 
vorgestellt, die im restlichen Verlauf der Arbeit immer wieder eine Rolle spielen.

\begin{bsp}[Zufallsspaziergang]
Ein einfacher stochastischer Prozess ist der \textit{Zufallsspaziergang} (random walk).
Die Position des Spaziergängers zur Zeit $t$ wird durch eine Zufallsvariable $X_t$ beschrieben. 
Man startet bei $X_0 = 0$. In jedem Zeitschritt bewegt sich der Spaziergänger entweder 
einen Schritt nach rechts oder nach links, jeweils mit Wahrscheinlichkeit $p$ bzw. $1-p$. 
Formal gilt:
$$
X_{t+1} = X_t + \xi_{t+1},
$$
wobei $\xi_{t+1}$ eine unabhängige Zufallsvariable ist mit
$$
\xi_{t+1} = 
\begin{cases} 
+1, & \text{mit Wahrscheinlichkeit } p, \\
-1, & \text{mit Wahrscheinlichkeit } 1-p.
\end{cases}
$$

\end{bsp}

\begin{bsp}[Binomialmodell]
Das Binomialmodell ist ein simples Modell einer Aktie und deren Preisentwicklung. 
Man beginnt mit einem Anfangspreis $S_0$. $S_1, S_2, \dots$ sind dann Messungen des Aktienpreises zu einem festen Intervall.
 Zu einer festen Wahrscheinlichkeit $0 \lt p \lt 1$ steigt die Aktie um den Faktor $d$, oder fällt mit der Wahrscheinlichkeit $1-p$ um den Faktor $u$. Also
$$S_{t+1} = \begin{cases} d \cdot S_t, \quad  p \\ u \cdot S_t, \quad 1-p\end{cases}.$$
Der Prozess ist multiplikativ und nicht additiv, da ein Aktienkurs nicht negativ werden kann.
\end{bsp}

\begin{bsp}[Brownsche Bewegung]
Die Brownsche Bewegung beschreibt ursprünglich die Bewegung eines Partikels in 
einer Flüssigkeit. Die Zufallsgröße ist hier die Position des Partikels.

Naturanaloge stochastische Prozesse erfüllen eine Stetigkeitsbedingung: 
Andernfalls würde sich der Partikel in der Flüssigkeit teleportieren. 
Konkret: Es bezeichne der $\text{Pfad}( X_t)$ die Menge von Realisierungen der Zufallsvariablen $X_t$. 
Im obigen Fall ist das der Weg des Partikels. Es wird gefordert, dass die Pfade fast-sicher stetig sind.

\end{bsp}

\subsection{Bedingter Erwartungswert und Filtrationen}

Im Folgenden wird stochastische Unabhängigkeit und der Erwartungswert im Bezug auf die Zeit untersucht. 
Dazu definiert man einen neuen Wahrscheinlichkeitsraum, 
der alle möglichen Verläufe des Prozesses vereint. 
Um nicht weiter zwischen zeit-stetigen und -diskreten Prozessen unterscheiden zu müssen, 
sei $I = \Bbb N_0$ im diskreten und $I = \Bbb R_{\ge 0}$ im stetigen Fall. 
Des weiteren sei $(\tilde \Omega, \mathcal A_t, \tilde P)$ ein Wahrscheinlichkeitsraum, 
und $(X_t)_{t \in I}$ ein eine Familie von Zufallsvariablen auf dem Wahrscheinlichkeitsraum, 
die einen stochastischen Prozess bilden. Im Folgenden wird der Prozess 
auf dem Produkt-Wahrscheinlichkeitsraum
$$(\Omega, \mathcal F, P) := \bigtimes_{t \in I}(\tilde \Omega, \mathcal A_t,\tilde P)$$
betrachtet.

Das Ziel des ersten Teils ist es, Aktienkurse mit einem stochastischen Prozess 
zu modellieren.

\begin{defi}[Adaptiertheit]
In der modernen Wahrscheinlichkeitsrechnung wird "Information über den
Wahrscheinlichkeitsraum $(\Omega, \mathcal F, P)$" als Teil-$\sigma$-Algebra 
von $\mathcal A$ verschlüsselt (Behrends). Dazu wird der Begriff der Filtration 
definiert: Eine Familie von $\sigma$-Algebren $\mathcal F_t, t \in I$ heißt *Filtration*, 
wenn $\mathcal F_s \subset \mathcal F_t$ für alle $s \lt t$ gilt. 
Der Prozess $(X_t)_{t \in I}$ heißt an die Filtration adaptiert, 
wenn $X_t$ $\mathcal F_t$-messbar ist für alle $t \in I$. Die Eigenschaft 
$\mathcal F_s \subset \mathcal F_t$ kann man wie folgt interpretieren: 
Die Größe des Datensatzes nimmt über die Zeit zu. Anstatt dem Produktraum, kann man 
die Zufallsvariablen $X_t$ somit auf den Räumen $(\Omega, \mathcal F_t, P)$ definieren, 
und setzt $\mathcal F = \lim_{t \to \infty} \mathcal F_t$.

\end{defi}

\begin{bsp}[Filtration des Binomialmodells]
Jeder Zeitschritt im Binomialmodell ist zunächst eine Zufallsvariable auf dem 
Wahrscheinlichkeitsraum $(\{W, B\}, \sigma(\{W, B\}), P)$ mit 
$P(\{W \}) = p$, $P(\{B \}) = 1-p$. Hier steht $W$ für den Kursanstieg und $B$ für Kursfall. Die Preisentwicklung bis zum Zeitpunkt $t_0$ fasst man als Produkt-Wahrscheinlichkeitsraum auf.
Betrachtet man ein dreistufiges Binomialmodell, ergibt sich die folgende Filtration: 
$$
\begin{aligned}
\mathcal F_0 &= \{\emptyset, \Omega\} \\
\mathcal F_1 &= \{\{WWW, WWB, \dots, WBB \}, \{ BWW, BWB, \dots, BBB \},\emptyset, \Omega \} \\ 
\mathcal F_2 &= \{ \{WWB, WWW \}, \{WBB, WBW \}, \{BWB, BWW \}, \{BBB, BBW \}, \\ &\{WWW, WWB, \dots, WBB \}, \{ BWW, BWB, \dots, BBB \}, \\ &\emptyset, \Omega \} \\
\mathcal F_3 &= \text{Pot}(\Omega)
\end{aligned}
$$
Der Prozess $S_t$ ist adaptiert: Im ersten Schritt kann der Kurs entweder steigen oder
fallen, alle weiteren Kursverläufe sind in dem Ereignis enthalten.

\end{bsp}

\begin{defi}[Bedingter Erwartungswert]

Der bedingte Erwartungswert ist wichtig für die Untersuchung stochastischer Prozesse,
insbesondere für unseren Anwendungsfall, da historische Entwicklungen in der Praxis 
meist bekannt sind. Beispiel Poker: Dem Spieler ist seine Hand, und die Karten auf dem 
Tisch bekannt. Daraus lässt sich die eigene Gewinnwahrscheinlichkeit berechnen.

Sei $B$ ein Ereignis in der Vergangenheit. Dann definiert man
den \textit{bedingten Erwartungswert} als $$E(X_t|B):=  \frac{1}{P(B)} \int_{B}^{} X_t(\omega) dP(\omega),$$
und für einen diskreten Wahrscheinlichkeitsraum reicht
$$E(X_t|B):= \frac{1}{P(B)}\sum_{b \in B} X_t(b) \cdot P(\{ b \})$$
wobei $P(B) \neq 0$ gilt. Sonst ist $E(X_t|B) :=0$. 
Mit dem bedingten Erwartungswert wird die Zufallsvariable $E(X_t)$.

Ebenfalls werden die Satze des totalen Erwartungswertes und des iterierten Erwartungswertprinzips bewiesen,
allerdings nur im diskreten Fall. Das ist ausreichend, da die meisten stetigen 
Prozesse als Grenzprozesse diskreter Prozesse aufgefasst werden können.
Dann werden Aussagen durch Grenzwertsätze auf den stetigen Fall übertragen. 

\end{defi}

\begin{bsp}[Rechenbeispiel]
Angenommen, die Kursentwicklung im dreistufigen Binomialmodell ist bis zum 
Zeitpunkt $t=1$ bekannt, nämlich ist der Preis um den Faktor $d$ gestiegen. 
Was kann man im Schritt $t=2$ erwarten? Hier ist $B=\{WWW, WWB, \dots, WBB \}$, $P(B)=p$, und gesucht ist $E(S_2|B)$. Aus der Definition folgt
$$
\begin{aligned}
E(S_2|B) &= \frac{1}{P(B)}\sum_{k \in B} S_2(k) \cdot P(\{ k \}) \\ &=\frac{1}{p}(S_2(WWW) \cdot P(WWW)+ \cdots + S_2(WBB) \cdot P(WBB)) \\
&= \frac{1}{p} (S_0 \cdot d^2 \cdot  p^2 + \cdots + S_0 \cdot d u \cdot p (1-p))
\end{aligned}
$$
Für den Startwert $S_0=10$ und $p=0.25$ so wie $d=\frac{1}{u}=2$ ergibt sich
$$E(S_2|B)=4 \cdot 10 \cdot \left( 2^2 \cdot 0.25^2 + 2^2 \cdot 0.25^2 + 1 \cdot 0.25\cdot 0.75 + 1 \cdot 0.25\cdot 0.75 \right)=35$$
Ist die bisherige Kursentwicklung bekannt, hier $S_0=10, S_1 = d \cdot S_0=20$, aber nicht $B$, müsste man zuerst $B:=S_1^{-1}(20)$ berechnen.
\end{bsp}

\begin{satz}[Totaler Erwartungswert]
Der totale Erwartungswert besagt, dass der Erwartungswert einer Zufallsvariablen
durch die Summe der bedingten Erwartungswerte über eine Partition des Wahrscheinlichkeitsraumes
berechnet werden kann. Sei also $(A_i)_{i \in I}$ eine Partition von $\Omega$,
dann gilt $$E(X_t) = \sum_{i \in I} E(X_t|A_i) \cdot P(A_i).$$
In dieser Arbeit wird nur der diskrete Fall betrachtet. \textit{Beweis.}
$$E(X_t) = \sum_{\omega \in \Omega} X_t(\omega) \cdot P(\{\omega\}) = \sum_{i \in I} \sum_{\omega \in A_i} X_t(\omega) \cdot P(\{\omega\}) = \sum_{i \in I} E(X_t|A_i) \cdot P(A_i).$$ $\square$

\end{satz}

\begin{satz}[Iterierter Erwartungswert]
Für eine Zufallsvariable $X$ und ein Ereignis $B$ mit $P(B) > 0$ gilt:
$$
E(E(X|B)) = E(X)
$$
\textit{Beweis.} Im diskreten Fall gilt:
$$
\begin{aligned}
E(E(X|B)) &= E\left(\frac{1}{P(B)} \sum_{b \in B} X(b) P(\{b\})\right) 
\\ &= \frac{1}{P(B)} \sum_{b \in B} X(b) P(\{b\}) \cdot P(B) 
\\ &= \sum_{b \in B} X(b) P(\{b\}) = E(X \cdot 1_B)
\end{aligned}
$$
Falls $B = \Omega$, folgt $E(X \cdot 1_B) = E(X)$. $\hfill \square$
\end{satz}

\subsection{Eigenschaften von stochastischen Prozessen}

\begin{defi}[Martingal]
Martingale sind Prozesse, die tendenziell weder steigen noch fallen, 
also "faire" Prozesse. Tendenziell heißt auf den Erwartungswert bezogen. 
Steigende Prozesse werden Supermartingale genannt, fallende Submartingale. 
Die mathematische Definitione erfolgt durch den Bedingten Erwartungswert: 
Ist ein "fairer" Kurs zum aktuellen Zeitpunkt $s$ auf einem bestimmten Wert, 
liegt der Erwartungswert zur Messzeit $t \gt s$ bei dem selben Wert.

Ein stochastischer Prozess $(X_t)_{t \in I}$ heißt \textit{Submartingal}, wenn 
$E(X_t|X_s=v) \le v$  für alle $s \lt t$, und alle $v \in \Bbb R$. $(X_t)_{t \in I}$
 heißt \textit{Supermartingal}, wenn  $(-X_t)_{t \in I}$ ein Submartingal ist, und \textit{Matringal}, 
 wenn er sowohl Supermartingal als auch Submartingal ist, also $E(X_t|X_s=v) = v$.
\end{defi}

\begin{lemma}[Martingaleigenschaft des Binomialmodells]
Das Binomialmodell genau dann ein Martingal, wenn $p=\frac{1-d}{u-d}$ gilt. 
\textit{Beweis}.
Zuerst wird der Fall $t = s + 1$ gezeigt:
Es gilt $S_{s+1}=S_s\,\xi_{s+1}$ für eine Zufallsvariable $\xi$ mit $\mathbb{P}(\xi_{s+1}=u)=p,\ \mathbb{P}(\xi_{s+1}=d)=1-p$. Es folgt
$$E(S_{s+1}|S_s=v)= v \cdot E(\xi_{s+1})=v(pu+(1-p)d).$$
Und setzt man die Matringaleigenschaft $E(S_{s+1}|S_s=v)=v$ ein, ergibt sich
$$pu+(1-p)d=1\\[6pt] \iff p=\frac{1-d}{u-d}.$$
Seien nun $s \lt t \in \Bbb N_0$ beliebig. Dann gilt $S_t=\xi_{t-1}\cdot \xi_{t-2}\cdots \xi_{s+1}\cdot S_s$. Da die $\xi_i$ unabhängig und identisch verteilt sind, gilt
$$E(S_t|S_s=v)=v \cdot \prod_{i=s+1}^{t-1} E(\xi_t)=v \cdot E(\xi_{s+1})^{t-s}=v \cdot (pu+(1-p)d)^{s-t}.$$
$E(S_t|S_s=v)=v$ ist wieder äquivalent zu $p=\frac{1-d}{u-d}$. $\hfill \square$
\end{lemma}

\begin{defi}[Markovprozess]
Ein stochastischer Prozess heißt Markovprozess, falls die 
Zufallsvariablen $X_t$ unabhängig von allen $X_s, s \lt t$ sind.
\end{defi}

\begin{bsp}
Das Binomialmodell und der Zufallsspaziergang sind Beispiele für Markovprozesse, da die zukünftige Position 
nur von der aktuellen Position und nicht von der gesamten Vergangenheit abhängt.
\end{bsp}

\section{Die Brownsche Bewegung}

Oft wird die Brownsche Bewegung (oder auch Wiener Prozess) axiomatisch definiert. In dieser Arbeit werden 
direkt kumulative Summen von Normalverteilungen betrachtet. Zuerst wird eine vereinfachte Darstellung des Prozesses eingeführt, 
die diskrete Brownsche Bewegung. Die Argumentation folgt der Darstellung von Behrends \cite{behrends} (2013, Kap. 5).

\subsection{Die diskrete Brownsche Bewegung}

\begin{defi}[Diskrete Brownsche Bewegung]
Die elementare Brownsche Bewegung sei ein stochastischer Prozess, 
der aus einer Folge von Zufallsvariablen $\xi_n, n \in \Bbb N_0$ besteht, wobei
$$\xi_n = \sum_{i=1}^n \eta_i, \quad \eta_i \sim \mathcal N(0,1).$$
Die Zufallsvariablen $\eta_i$ sind unabhängig und identisch verteilt.
Nun wird eine stetige Zeitentwicklung durch lineare Interpolation eingeführt:
$$b^{(1)}(t) := \xi_{\lfloor t \rfloor} + (t - \lfloor t \rfloor)(\xi_{\lfloor t \rfloor + 1} - \xi_{\lfloor t \rfloor}), \quad t \geq 0.$$
Die Funktion $b^{(1)}(t)$ wird diskrete Brownsche Bewegung erster Ordnung genannt.
Der Name rührt daher, dass in eine Zeiteinheit genau eine Normalverteilung einbezogen wird.
Im Allgemeinen wird die diskrete Brownsche Bewegung $b^{(N)}(t)$ $N$-ter Ordnung definiert als
$$b^{(N)}(t) := \frac{1}{\sqrt{N}} \left ( \xi_{\lfloor Nt \rfloor} + (Nt - \lfloor Nt \rfloor)(\xi_{\lfloor Nt \rfloor + 1} - \xi_{\lfloor Nt \rfloor}) \right ), \quad t \geq 0.$$
Hierbei werden in eine Zeiteinheit $N$ Normalverteilungen einbezogen. Der Faktor $1/\sqrt{N}$ dient dazu, die Varianzen der Normalverteilungen zu normieren.

\end{defi}

\begin{bsp}[Visualisierung der diskreten Brownschen Bewegung]
Das folgende R-Programm (Ausschnitt) generiert eine diskrete Brownsche Bewegung $N$-ter Ordnung.

\begin{lstlisting}
  n_points <- N * T_max
  eta <- rnorm(n_points, mean=0, sd=1)
  xi <- c(0, cumsum(eta))
  t_grid <- seq(0, T_max, length.out=steps)
  k <- floor(N * t_grid)
  frac <- N * t_grid - k
  vals <- xi[k+1] + frac * (xi[k+2] - xi[k+1])
  vals <- vals / sqrt(N)
\end{lstlisting}
Für verschiedene Werte von $N$ ergeben sich die folgenden Grafiken:

\begin{figure}[H]
    \centering
    \includegraphics[width=0.9\textwidth]{images/disrete_bb.png}
    \caption{Diskrete Brownsche Bewegung erster, zehnter, fünfzigster und zweihundertster Ordnung}
    \label{fig:brownian}
\end{figure}

\end{bsp}

\begin{lemma}[Erwartungswert der diskreten Brownschen Bewegung]
Es gilt
$$
E(b^{(N)}(t)) = 0.
$$
\textit{Beweis}.
Da $b^{(N)}(t)$ eine lineare Kombination von zentrierten Normalverteilungen ist, gilt
$$
E(b^{(N)}(t)) = \frac{1}{\sqrt{N}} \left( E(\xi_{\lfloor Nt \rfloor}) + (Nt - \lfloor Nt \rfloor) E(\xi_{\lfloor Nt \rfloor + 1} - \xi_{\lfloor Nt \rfloor}) \right).
$$
Da $E(\xi_n) = 0$ für alle $n$ und $E(\xi_{n+1} - \xi_n) = E(\eta_{n+1}) = 0$, folgt
$$
E(b^{(N)}(t)) = 0.
$$
\qed
\end{lemma}

\begin{lemma}[Varianz der diskreten Brownschen Bewegung]
Es gilt
$$
V(b^{(N)}(t)) = \frac{\lfloor Nt \rfloor}{N} + \frac{(Nt - \lfloor Nt \rfloor)^2}{N}.
$$
\textit{Beweis}. Aus der Unabhängigkeit der Inkremente folgt
$$
\begin{aligned}
V(b^{(N)}(t)) &= \frac{1}{N} \left ( V(\xi_{\lfloor Nt \rfloor}) + (Nt - \lfloor Nt \rfloor)^2 \cdot V(\xi_{\lfloor Nt \rfloor + 1} - \xi_{\lfloor Nt \rfloor}) \right ) 
\\ &= \frac{1}{N} (\lfloor Nt \rfloor + (Nt - \lfloor Nt \rfloor)^2)  
\\ &= \frac{\lfloor Nt \rfloor}{N} + \frac{(Nt - \lfloor Nt \rfloor)^2}{N}.
\end{aligned}
$$
\qed
\\
Im Grenzübergang $N \to \infty$ konvergiert $\frac{\lfloor Nt \rfloor}{N} \to t$ und $\frac{(Nt - \lfloor Nt \rfloor)^2}{N} \to 0$.
\end{lemma}

\subsection{Die Brownsche Bewegung als Grenzprozess}
Nun wird der Grenzprozess $N \to \infty$ betrachtet. Intuitiv wird die Zeit immer feiner aufgelöst,
und es werden immer mehr Normalverteilungen in eine Zeiteinheit einbezogen. Vorerst ist jedoch unklar, 
ob der Grenzprozess überhaupt existiert. Um die Konvergenz des Prozesses zu zeigen, wird die Verteilungskonvergenz 
untersucht. Konvergenz der einzelnen Zeitpunkte (oder endlich-dimensionalen Vektoren) reicht nicht 
für Konvergenz der Prozesse als Funktionen. Daher wird die Konvergenz in 3 Schritten untersucht:
\begin{enumerate}
  \item Verteilungskonvergenz der einzelnen Zeitpunkte
  \item Verteilungskonvergenz von endlich-dimensionalen Vektoren und die Kovarianzen (Hieraus folgt bereits die Selbstähnlichkeit und bedingte Verteilung so wie die Martingal-Eigenschaft)
  \item Stetigkeit der Pfade des Grenzprozesses
\end{enumerate}
Die Verteilungskonvergenz im Funktionenraum $C[0, 1]$ der stetigen Funktionen $f : [0, 1] \to \Bbb R$ wird in dieser Arbeit nicht behandelt.
Diese kann mit dem Satz von Donsker (\cite{henze2022asymptotische}, S. 270) nachgewiesen werden.

\begin{lemma}[Verteilungskonvergenz zu einzelnen Zeitpunkten]
Es existiert ein stochastischer Prozess $W_t, t \geq 0$, so dass für jedes $t$ die Verteilung von $b^{(N)}(t)$ gegen die Verteilung von $W_t$ konvergiert, wenn $N \to \infty$.
\textit{Beweis.}
Da 
$$b^{(N)}(t) = \frac{1}{\sqrt{N}}\big(\xi_k+\alpha(\xi_{k+1}-\xi_k)\big)
=\frac{1}{\sqrt{N}}\big(\xi_k+\alpha\,\eta_{k+1}\big).
$$
ist $b^{(N)}(t)$ eine lineare Kombination von Normalverteilungen, und daher widerum 
normalverteilt. Erwartungswert und Varianz wurden bereits berechnet:
$$
E(b^{(N)}(t)) = 0, \quad V(b^{(N)}(t)) = \sigma_N^2(t) = \frac{1}{N}\big(k+\alpha^2\big),
$$
wobei $k=\lfloor Nt \rfloor$ und $\alpha=Nt-k\in[0,1)$. Damit gilt
$$
b^{(N)}(t) \sim N\left(0,\frac{1}{N}\big(k+\alpha^2\big)\right).
$$
Die Verteilungsfunktion ist gegeben durch
$$
F_N(x)=P\big(b^{(N)}(t)\le x\big)=\Phi\!\left(\frac{x}{\sigma_N(t)}\right),
$$
wobei $\Phi$ die Standardnormalverteilungsfunktion ist. Aus $\sigma_N^2(t)\to t$ folgt $\sigma_N(t)\to \sqrt{t}$ und wegen der Stetigkeit von $\Phi$ daher
$$
F_N(x)=\Phi\!\left(\frac{x}{\sigma_N(t)}\right)\; \xrightarrow[N \to \infty]{\mathrm{pktw}} \;\Phi\!\left(\frac{x}{\sqrt{t}}\right).
$$
Dies ist die Verteilungsfunktion von $N(0,t)$. Definiere $W_t :\sim \mathcal N(0,t)$. Somit konvergiert für jedes feste $t$ die Verteilung von $b^{(N)}(t)$ gegen die von $W_t$. 
\qed
\end{lemma}

\begin{satz}[endlich-dimensionale Verteilungskonvergenz und Kovarianzstruktur der Brownschen Bewegung]
Aus der Verteilungskonvergenz der einzelnen Zeitpunkte kann man noch keinen sinnvollen Grenzprozess folgern.
Die Zeitpunkte $W_t$ sind zwar normalverteilt, aber der Prozess könnte trotzdem sprunghaft sein.
Nun wird die Verteilungskonvergenz der endlich-dimensionalen Vektoren gezeigt, beziehungsweise
die Kovarianz der Realisierungen bei benachbarten Zeitpunkten untersucht. Für $t$ und $s$ gilt 
$$
\mathrm{Cov}(W_s, W_t) = \min(s,t).
$$
Sei $0\le t_1\le\cdots\le t_k$ eine Zerlegung. Dann gilt
$$
\big(b^{(N)}(t_1), \dots , b^{(N)}(t_k) \big )\;\xrightarrow[N \to \infty]{\mathrm{d}}\;\big(W_{t_1},\dots,W_{t_k}\big).
$$
\textit{Beweis.}\footnote{angelehnt an Henze, 2023 \cite{henze} S. 225f.}
Für jedes $N$ ist der Vektor $\big(b^{(N)}(t_1),\dots,b^{(N)}(t_k)\big)$ gemeinsam normalverteilt,
denn $b^{(N)}(t)$ ist eine lineare Kombination der i.i.d. standardnormalen $(\eta_i)$.

Es genügt, Mittelwerte und Kovarianzen zu kontrollieren:
Die Mittelwerte sind Null. Für $s,t\ge0$ mit $k=\lfloor Nt\rfloor$, $\alpha=Nt-k$, 
$\ell=\lfloor Ns\rfloor$, $\beta=Ns-\ell$ gilt mit $\xi_n=\sum_{i=1}^n\eta_i$:
$$
b^{(N)}(t)=\tfrac1{\sqrt N}\big(\xi_k+\alpha\eta_{k+1}\big),\qquad
b^{(N)}(s)=\tfrac1{\sqrt N}\big(\xi_\ell+\beta\eta_{\ell+1}\big).
$$
Unabhängigkeit der $\eta_i$ ($E(\eta_i)=0$, $V(\eta_i)=1$) liefert
$$
\begin{aligned}
\mathrm{Cov}\!\big(b^{(N)}(s),b^{(N)}(t)\big) &= E \left [ b^{(N)}(s) b^{(N)}(t) \right ] - \underbrace{E(b^{(N)}(s)) \cdot b^{(N)}(t)}_{=0}
\\ &= \frac{1}{N} E \left [ (\xi_k + \alpha \eta_{k+1}) (\xi_\ell + \alpha \eta_{\ell+1}) \right ] 
\\ &= \frac{1}{N} \left [ E(\xi_\ell \xi_k) + \alpha E(\xi_\ell \eta_{k+1}) + \beta E(\eta_{\ell + 1} \xi_k) + \alpha \beta E(\eta_{\ell + 1} \eta_{k+1}) \right ]
\end{aligned}
$$
Da die Erwartungswerte der $\eta$s gleich Null sind und die $\eta$s unabhängig sind, können die Terme weiter reduziert werden.
$$
\begin{aligned}
E(\xi_\ell \xi_k) &= \sum_{i=1}^\ell \sum_{j=1}^k E(\eta_i \eta_j) \begin{cases} = E(\eta_i)\cdot E(\eta_j) = 0, \quad &i \neq j \\ = E(\eta_i^2) = 1, \quad &i = j \end{cases}
\\ &= \sum_{i=1}^\ell \sum_{j=1}^k \delta_{i, j} = \min(\ell, k)
\end{aligned}
$$
Analog folgt für die anderen Terme z. B.
$$\alpha E(\xi_\ell \eta_{k+1}) = \alpha \sum_{i=1}^\ell E(\eta_i \eta_{k+1}) = \alpha \mathbf 1_{\{k+1\le \ell\}},$$
und so weiter. Insgesamt ergibt sich
$$
\mathrm{Cov}\!\big(b^{(N)}(s),b^{(N)}(t)\big) = \frac1N\Big(\min(\ell,k)+\alpha\,\mathbf 1_{\{k+1\le \ell\}}+\beta\,\mathbf 1_{\{\ell+1\le k\}}+\alpha\beta\,\mathbf 1_{\{\ell=k\}}\Big)
$$
Im Grenzübergang gilt
$$
\mathrm{Cov}\!\big(b^{(N)}(s),b^{(N)}(t)\big)
=\frac{\min(\ell,k)}{N}+O\!\left(\frac1N\right)\xrightarrow[N\to\infty]{}\min(s,t).
$$
Folglich konvergiert die Kovarianzmatrix der Vektoren gegen 
$\Sigma=(\min(t_i,t_j))_{i,j}$.
Mit der Cramér–Wold-Technik (\ref{satz:cramer_wold}) reicht es, lineare Formen zu betrachten. Sei also $a=(a_1,\dots,a_k)^\top\in\Bbb R^k$ und
$$
Y_N:=\sum_{i=1}^k a_i\,b^{(N)}(t_i),\qquad
Y:=\sum_{i=1}^k a_i\,W_{t_i},
$$
wobei
$$
V(Y_N)=\sum_{i,j=1}^k a_i a_j\,\mathrm{Cov}\!\big(b^{(N)}(t_i),b^{(N)}(t_j)\big),
$$
und damit
$$
V(Y_N)\xrightarrow[N\to\infty]{}\sum_{i,j=1}^k a_i a_j\,\min(t_i,t_j)=:\sigma^2(a).
$$
Jedes $Y_N$ ist eine lineare Kombination unabhängiger $\eta_i$. Mit dem Zentralen Grenzwertsatz von Lindeberg-Feller (\ref{satz:lindeberg_feller}) wird die Grenzverteilung von $Y_N$ bestimmt.
Da die Summanden zentriert und von endlicher Varianz sind, bleibt die Lindeberg-Bedingung zu zeigen. Zerlege
$$
Y_N=\sum_{i=1}^k a_i b^{(N)}(t_i)=\sum_{j=1}^{M_N} c_{N,j}\,\eta_j,
$$
wobei $M_N=\lfloor Nt_k\rfloor+1$ und die Koeffizienten $c_{N,j}$ aus der Definition von $b^{(N)}(t_i)$ stammen.
Aus der Darstellung folgt, dass $|c_{N,j}|\le C/\sqrt N$ für ein von $N$ unabhängiges $C>0$ gilt: Die Definition
$$
\begin{aligned}
b^{(N)}(t) &\overset{\mathrm{def}}= \frac{1}{\sqrt{N}} \left ( \xi_{\lfloor Nt \rfloor} + (Nt - \lfloor Nt \rfloor)(\xi_{\lfloor Nt \rfloor + 1} - \xi_{\lfloor Nt \rfloor}) \right )
\\ &= \frac{1}{\sqrt{N}} \left ((Nt - \lfloor Nt \rfloor) \cdot \eta_{n+1} + \sum_{i=1}^n \eta_i \right ),\quad n = \lfloor Nt \rfloor
\end{aligned}
$$
liefert
$$
C = k \cdot \max_{i=1, \dots, k} \vert a_i \vert.
$$
Damit erhält man für jedes $\varepsilon>0$
$$
\mathbf 1_{\{|c_{N,j}\eta_j|>\varepsilon\}}
= \mathbf 1_{\{|\eta_j|>\varepsilon / \vert c_{N,j} \vert \}}
\le \mathbf 1_{\{|\eta_j|>\varepsilon\sqrt N/C\}}.
$$
Also
$$
L_N(\varepsilon) = \sum_{j=1}^{M_N} E \left (c_{N,j}^2\eta_j^2\,\mathbf 1_{\{|c_{N,j}\eta_j|>\varepsilon\}} \right )
\le \Big(\sum_{j=1}^{M_N} c_{N,j}^2 \Big) E \left [ \eta_1^2\mathbf 1_{\{|\eta_1|>\varepsilon\sqrt N/C\}} \right ].
$$
Die Summe der Quadrate $\sum_j c_{N,j}^2$ ist beschränkt, da sie die Varianz von $Y_N$ ist und gegen
$\sigma^2(a)$ konvergiert. Andererseits gilt wegen $\eta_1\in L^2$ und $\varepsilon\sqrt N/C \to \infty$
$$
E \left [ \eta_1^2\mathbf 1_{\{|\eta_1|>\varepsilon\sqrt N/C\}} \right ] \xrightarrow[N\to\infty]{}0.
$$
Damit ist die Lindeberg-Bedingung erfüllt und es folgt
$$
Y_N \xrightarrow{d} Y \sim \mathcal N(0,\sigma^2(a)).
$$
Insgesamt konvergiert für jedes $a$ die Verteilung von $a^\top(b^{(N)}(t_1),\dots,b^{(N)}(t_k))$ gegen die von $a^\top(W_{t_1},\dots,W_{t_k})$.
Nach Cramér–Wold folgt die behauptete Verteilungskonvergenz des Vektors. \qed
\end{satz}

\begin{satz}[Selbstähnlichkeit und bedingte Verteilung der Brownschen Bewegung]
Für jedes $c > 0$ gilt
$$
W_{ct}(\omega) \overset{d}{=} \sqrt{c}\, W_t(\omega) \quad \text{für alle } t \ge 0,
$$
für alle $s < t$ gilt
$$
W_t - W_s \sim \mathcal N(0, t-s),
$$
und die bedingte Verteilung von $W_t$ gegeben $W_s$ ist
$$
W_t \mid W_s \sim N\big(W_s, t-s\big).
$$
\textit{Beweis der ersten Behauptung (Selbstähnlichkeit).} \\
Für $t \ge 0$ gilt $W_{ct} \sim \mathcal N(0, ct)$ und $W_t \sim \mathcal N(0, t)$. Somit gilt
$$
\sqrt{c}\, W_t \sim \mathcal N(0, c t).
$$
Da beide Normalverteilungen denselben Mittelwert $0$ und dieselbe Varianz $ct$ haben, folgt
$$
W_{ct} \stackrel{d}{=} \sqrt{c}\, W_t \quad \text{für jedes } t \ge 0.
$$  
\qed  \\
\textit{Beweis der zweiten Behauptung (Inkremente).} \\
Für $s < t$ gilt
$$
W_t - W_s = \lim_{N \to \infty} \big( b^{(N)}(t) - b^{(N)}(s) \big),
$$
wobei $b^{(N)}$ diskrete Approximierungen sind. Jedes Inkrement $b^{(N)}(t) - b^{(N)}(s)$ ist normalverteilt mit Erwartungswert $0$ und Varianz $t-s$. Durch Grenzwertbildung folgt
$$
W_t - W_s \sim \mathcal N(0, t-s).
$$
\qed  \\
\textit{Beweis der dritten Behauptung (Bedingte Verteilung).} \\
Betrachte $(W_s, W_t)^T$, das multivariat normal verteilt ist mit
$$
E(W_s) = E(W_t) = 0, \quad
\mathrm{Cov}(W_s, W_t) = \min(s, t) = s.
$$  
Die Kovarianzmatrix lautet somit
$$
\Sigma = \begin{pmatrix} s & s \\ s & t \end{pmatrix}.
$$
Für die bedingte Verteilung ergibt die Standardformel der multivariaten Normalverteilung (s. \cite{soch_conditional_2020})
$$
\begin{aligned}
E(W_t \mid W_s) &= E(W_t) + \mathrm{Cov}(W_t, W_s)\,V(W_s)^{-1} (W_s - E(W_s)) \\
&= 0 + \frac{s}{s} W_s = W_s, \\
V(W_t \mid W_s) &= V(W_t) - \frac{\mathrm{Cov}(W_t, W_s)^2}{V(W_s)} \\
&= t - \frac{s^2}{s} = t - s.
\end{aligned}
$$
Insgesamt folgt
$$
W_t \mid W_s \sim \mathcal N(W_s,\, t-s).
$$ \qed
\end{satz}

\begin{bem}[Visualisierung der bedingten Verteilung]
Um die Wichtigkeit der Kovarianz-Struktur zu verdeutlichen, wird in der folgenden Visualisierung gezeigt, wie ein
stochastischer Prozess ohne Kovarianz der einzelnen Zufallsvariablen aussehen würde.
In Blau werden die (bedingten) Verteilungen der Zufallsvariablen, und in Rot jeweils Realisierungen des Prozesses dargestellt.

Dass die bedingte Verteilung um den vorigen Wert zentriert ist, kann man als ein schwaches Stetigkeits-Kriterium interpretieren.
Ausgehend von der Kovarianz-Struktur kann man die Stetigkeit mit dem Stetigkeitssatz von Kolmogorov (\cite{behrends} S. 78f.) folgern.
In dieser Arbeit wird jedoch ein anderer Beweisansatz genutzt.

\begin{figure}[H]
  \centering
  \begin{minipage}{0.48\textwidth}
    \includegraphics[width=\textwidth]{images/bb_without_cov.png}
    \caption{Folge von unabhängigen Normalverteilten Zufallsvariablen. Die Realisierungen sind chaotisch.}
    \label{fig:bb_without_cov}
  \end{minipage}\hfill
  \begin{minipage}{0.48\textwidth}
    \includegraphics[width=\textwidth]{images/bb_with_cov.png}
    \caption{Visualierung der bedingten Verteilung der diskreten Brownschen Bewegung.}
    \label{fig:bb_with_cov}
  \end{minipage}
\end{figure}
\end{bem}

\begin{korr}[Martingal-Eigenschaft der Brownschen Bewegung]
Die Brownsche Bewegung $W_t$ ist ein Martingal.
\textit{Beweis.} Aus der bedingten Verteilung folgt
$$
E(W_t | W_s = v) = v.
$$
Damit ist $W_t$ ein Martingal. \qed
\end{korr}  

\begin{satz}[Stetigkeit der Pfade]
Der Pfad $t \mapsto W_t(\omega)$ ist fast sicher stetig.
\textit{Beweis.}
Für den Beweis wird eine neue Funktionen-Folge $\hat W_t^{(n)}(\omega), n \in \Bbb N$
definiert, wobei $\hat W_t^{(n)}(\omega)$ die lineare Interpolation der Werte $W_{k/2^n}(\omega), k=0,1,2,\ldots$ ist. 
Ohne Beschränkung der Allgemeinheit wird das Intervall $[0,1]$ betrachtet. Für $n \in \Bbb N$ und $k=0,\dots,2^n-1$ setze
$$I_{n,k}:=\big[k2^{-n},(k+1)2^{-n}\big].$$
Da die $I_{n,k}$ eine Zerlegung von $[0,1]$ bilden, ist
$$
\begin{aligned}
M_n &:=\sup_{t\in[0,1]}\big|\hat W^{(n+1)}_t-\hat W^{(n)}_t\big| 
\\ &= \max_{0\le k<2^n} \sup_{t\in I_{n,k}} \big|\hat W^{(n+1)}_t-\hat W^{(n)}_t\big| =\max_{0\le k<2^n}|Z_{n,k}|
\end{aligned}
$$
Für die unabhängigen Inkremente $Z_{n,k}$ mit
$$
Z_{n,k}\sim N\!\big(0,2^{-(n+2)}\big).
$$
Für eine Normalverteilung $Z\sim \mathcal N(0,\sigma^2)$ und jedes $\varepsilon>0$ gilt (\cite{boucheron_concentration_2013} S. 2)
$$
P(|Z| >\varepsilon) \le 2\exp\!\Big(-\frac{\varepsilon^2}{2\sigma^2}\Big).
$$
Mit der Schranke folgt
$$
P(M_n>\varepsilon) \le \sum_{k=0}^{2^n-1} P(|Z_{n,k}|>\varepsilon)
\le 2^n\cdot 2\exp\!\Big(-\frac{\varepsilon^2}{2\sigma_n^2}\Big)
= 2^{\,n+1}\exp\!\Big(-\frac{\varepsilon^2}{2\sigma_n^2}\Big).
$$
Man wählt nun eine spezifische Folge $\varepsilon_n$ so, dass sich aus der rechten Seite eine geometrische Reihe ergibt. Setze
$$
\varepsilon_n := \sqrt{2(n+1)}\,\sigma_n.
$$
Dann gilt
$$
\frac{\varepsilon_n^2}{2\sigma_n^2}=n+1,
$$
und damit
$$
P(M_n>\varepsilon_n)\le 2^{\,n+1} e^{-(n+1)} = \big(2e^{-1}\big)^{\,n+1}.
$$
Da $2e^{-1}<1$ ist, ist die Folge auf der rechten Seite geometrisch und insbesondere summierbar. Folglich
$$
\sum_{n=1}^\infty P(M_n>\varepsilon_n) < \infty.
$$
Mit dem Reihenkriterium für fast sichere Konvergenz (\ref{lemma:reihenkriterium}) folgt, dass $M_n\to 0$ fast sicher.
Für $m \gt n \ge N$ beliebig gilt
$$\sup_{t}|\hat W^{(m)}_t - \hat W^{(n)}_t| \leq \sum_{k=n}^{m-1} M_k. \underset{N \to \infty} \longrightarrow 0$$
Weil jedes $\big(\hat W^{(n)}\big)_{n\in\Bbb N}$ eine stückweise lineare Funktion ist, folgt
dass $\big(\hat W^{(n)}\big)_{n\in\Bbb N}$ fast sicher eine Cauchy-Folge in $\Vert \cdot \Vert_{\infty}$ ist.
$\big(\hat W^{(n)}\big)_{n\in\Bbb N}$ konvergiert also gleichmäßig gegen den Grenzpfad $\widetilde W$, der stetig ist. \qed

\end{satz}

\newpage

\section{Die geometrische brownsche Bewegung}

Die geometrische brownsche Bewegung ist eine Erweiterung der Brownschen Bewegung. 
Sie eignet zur Modellierung von Aktienkursen, da sie im Gegensatz zur klassischen 
Brownschen Bewegung stets positive Werte annimmt.


\subsection{Die geometrische Brownschen Bewegung}

Das \textit{Binomialmodell} beschreibt den Aktienkurs $S_t$ in diskreter Zeit: In jedem Zeitschritt ändert sich der Kurs multiplikativ um einen Zufallsfaktor. Es gilt
$$
S_{k+1} = S_k \,(1 + X_{k+1}), \qquad k = 0,1,\dots,n-1,
$$
für eine Zufallsvariable $X_{k+1}$ die die relative Kursänderung im Schritt $k+1$ repräsentiert. Um eine kontinuierliche Zeitentwicklung zu modellieren, 
setzt man\footnote{Dies ist ein Schritt des Euler-Maruyama-Verfahren zur numerischen Lösung der 
stochastischen Differentialgleichung der geometrischen Brownschen Bewegung. Vgl. Kapitel 8 oder Bärwolff und Tischendorf \cite{Baerwolff2025}.}
$$
X_{k+1} = \mu  \Delta t + \sigma \sqrt{\Delta t}\varepsilon_{k+1},
$$
mit $\mu$ als Erwartungswert der Rendite (oder auch Drift), $\sigma$ als Volatilität, und $\varepsilon_{k+1}$ unabhängig, identisch verteilten Zufallsvariablen mit Erwartungswert $0$ und Varianz $1$. 
(An dieser Stelle ist unwichtig, wie die $\varepsilon_{k+1}$ verteilt sind, es reicht, dass sie diese Momente besitzen. 
Die Information über die Verteilung wird in einem Grenzübergang verloren gehen.)
Das bildet die Modellannahme für den Rest dieser Arbeit.

Im Folgenden wird eine explizite Formel für $S_t$ bewiesen, nämlich
$$S_T = S_0 \exp\!\Big( (\mu - \tfrac12 \sigma^2)T + \sigma W_T \Big),$$
wobei $W_T$ eine brownsche Bewegung ist und $T=n \cdot \Delta t$. \textit{Beweis.}
Man betrachtet den Logarithmus der $S_k$: Nach $n$ Schritten ist der Aktienkurs
$$
S_n = S_0 \prod_{j=1}^n (1 + X_j).
$$
Durch den Logarithmus erhält man
\begin{equation} \label{eq:log_sn}
\log S_n = \log S_0 + \sum_{j=1}^n \log(1+X_j).
\end{equation}
Als nächstes wird die Taylor-Entwicklung der Terme $\log(1+X_j)$ betrachtet. Die $k$-te Ableitung lautet
$$\log(1+x)^{(k)}=(-1)^{k+1} \frac{(k-1)!}{(1+x)^k}.$$
Setzt man diese in die Taylor-Formel ein ergibt sich
$$\log(1+x) = \sum_{k=0}^{\infty} \frac{(\log(1+\cdot)^{(k)})(0)}{k!}(x-0)^k= \sum_{k=1}^\infty(-1)^{k+1} \frac{x^k}{k}$$
Da $X_j \in O(\sqrt{\Delta t})$ ist, reicht die Taylor-Entwicklung bis zum quadratischen Term: Es gilt
$$
X_j^k = (\mu \Delta t + \sigma \sqrt{\Delta t} \varepsilon_j)^k.
$$
Durch den Binomischen Lehrsatz ergibt sich:
$$
X_j^k = \sum_{m=0}^k \binom{k}{m} (\mu \Delta t)^{k-m} (\sigma \sqrt{\Delta t} \varepsilon_j)^m.
$$
Die Terme enthalten Potenzen von $\Delta t$ in der Form $(\Delta t)^{(k-m) + m/2} = (\Delta t)^{k-m/2}$. Für $k \geq 3$ (und $m=0,\dots k$) ist $(\Delta t)^{k-m/2}$ von 
höherer Ordnung als $\Delta t$ oder $\sqrt{\Delta t}$ und verschwindet daher im Grenzübergang $\Delta t \to 0$ bzw. $n \to \infty$. Der Grenzübergang 
von diskreter zu kontinuierlicher Zeit führt eben dazu, dass $n$ und $\Delta t$ gleichzeitig gegen $\infty$ bzw. $0$ gehen, wobei $n \Delta t = T$ konstant bleibt.
Daher verschwinden in der Summe in Formel \ref{eq:log_sn} genau die Terme, in denen $\Delta t$ einen Exponenten größer $1$ hat.
$$
\log(1+X_j) \approx X_j - \tfrac12 X_j^2.
$$
Der quadratische Term wird nun ausmultipliziert und in der selben Weise abgeschätzt:
$$
\tfrac12 X_j^2 \approx \tfrac12 \sigma^2 \Delta t \,\varepsilon_j^2.
$$
Hier verschwinden die Terme $\mu (\Delta t)^2 \in o((\Delta t)^{2})$ und $2 \mu (\Delta t) \sigma \sqrt{\Delta t} \varepsilon_{k+1} \in o((\Delta t)^{3/2})$ wieder im Limes. Zwischenfazit:
$$
\begin{aligned}
\log S_n &\approx \log S_0 + \sum_{j=1}^n\left( \underbrace{\mu \Delta t + \sigma\sqrt{\Delta t} \varepsilon_j}_{X_j} - \underbrace{\frac{1}{2} \sigma^2 \Delta t \varepsilon_j}_{-\frac{1}{2} X_j^2} \right)
\\ &= \log S_0 + \mu T + \sigma\sqrt{\Delta t} \sum_{j=1}^{n} \varepsilon_j - \frac{1}{2} \sigma^2 \sum_{j=1}^{n} \Delta t \varepsilon_j^2 
\end{aligned}
$$
Im Folgenden wird der Grenzübergang $n \longrightarrow \infty$ bzw. $\Delta t \longrightarrow 0$ durchgeführt. 
Die erste Summe konvergiert nach dem Zentralen Grenzwertsatz (Verteilungskonvergenz):
$$
\sigma \sqrt{\Delta t} \sum_{j=1}^n \varepsilon_j = \sigma \frac{\sqrt{T}}{\sqrt{N}} \sum_{j=1}^n \varepsilon_j  \xrightarrow[n \to \infty]{\mathrm{d}} \xi \sim N(0, \sigma^2 T),
$$
also gegen eine Zufallsvariable $\xi$ die normalverteilt ist, mit Varianz $\sigma^2 T$.  $\xi = \sigma W_T$ ist eine Lösung, 
weil $W_T \sim N(0, T)$ eine brownsche Bewegung zur Zeit $T$ ist. 
Da $E(\varepsilon_j)=0$ und $V(\varepsilon_j)=1$ und damit $E(\varepsilon_j^2) = 1$ gilt, folgt im Grenzübergang für die zweite Summe nach dem Gesetz der großen Zahlen
$$
\frac{1}{2} \sigma^2 \sum_{j=1}^{n} \Delta t \varepsilon_j^2  \xrightarrow[n \to \infty]{\mathrm{f.s.}} \tfrac12 \sigma^2 T.
$$
Damit ergibt sich im Grenzübergang $n \to \infty$:
$$
\log S_T \overset{d} = \log S_0 + \big(\mu - \tfrac12 \sigma^2\big)T + \sigma W_T.
$$Exponentiell geschrieben erhält man die \textit{geometrische brownsche Bewegung}:
$$
S_T \overset d = S_0 \exp\!\Big( (\mu - \tfrac12 \sigma^2)T + \sigma W_T \Big).
$$
\qed
\subsection{Die logarithmische Normalverteilung}

Eine Zufallsvariable $X$ heißt log-Normalverteilt mit Varianz $\sigma^2$ und Erwartungswert $\mu$, 
falls die Zufallsvariable $Y := \log(X)$ normalverteilt ist mit $Y \sim N(\mu, \sigma^2)$.
$S_T$ ist somit log-normalverteilt:
$$\log S_T \overset{d} = \log S_0 + \big(\mu - \tfrac12 \sigma^2\big)T + \sigma W_T.$$
ergibt
$$\log S_T \sim N\left( \log S_0 + \left( \mu - \frac{1}{2} \sigma^2 \right)T , \sigma^2 T\right)$$
Erwartungswert und Varianz eines Kurses zum Zeitpunkt $T$ sind damit
$$E(S_T)=\log S_0 + (\mu - \tfrac12 \sigma^2)T, \quad V(S_T)=\sigma^2 T.$$


\section{Anwendungen auf Zeitreihen}

\subsection{Kalibrierung}
Aus einem Datensatz lassen sich die Parameter $\mu$ und $\sigma$ der
geometrischen Brownschen Bewegung schätzen. 
Für reale Werte ist $\Delta t \gt 0$ und $n$ ist die (endliche) Anzahl von Datenpunkten. 
Zur Schätzung von $\mu$ und $\sigma$ werden die log-Rendite
$$r_j := \log S_j - \log S_{j-1}= \big(\mu - \tfrac12 \sigma^2\big)\Delta t + \sigma (W_j - W_{j-1})$$
genutzt. Da $W_j - W_{j-1} \sim N(0, \Delta t)$ folgt
$$r_j \sim N((\mu - \tfrac12 \sigma^2)\Delta t, \sqrt{\sigma} \Delta t).$$
Man berechnet also die log-Rendite $\hat r_j$ des Datensatzes, 
und davon den empirischen Erwartungswert $m$ (den Durchschnitt) und die empirische Varianz $s^2$. 
Dann folgt $$\sigma \approx s,\quad \mu \approx m + \frac{1}{2} s^2.$$
Die Schätzung der Parameter kann in R wie folgt durchgeführt werden:

\begin{lstlisting}
log_returns <- diff(log(dax$Price)) # tägliche Werte
sigma <- sd(log_returns)
mu <- mean(log_returns) + 0.5 * sigma^2
\end{lstlisting}

\subsection{Back-Tests}

\begin{lemma}[Metriken für Zeitreihenschätzungen]
\end{lemma}


\subsection{Bootstrap-Verfahren zur Kalibrierung}

\subsection{Berechnung von Konfidenzintervallen}

Da $S_T$ log-normalverteilt ist, reicht es ein Konfidenzintervall für die log-Normalverteilung
zu berechnen.
Sei $X \sim N(\mu, \sigma^2)$ eine normalverteilte Zufallsvariable.
Dann ist $Y := e^X$ log-normalverteilt mit Parametern $\mu$ und $\sigma^2$.
Ein zweiseitiges Konfidenzintervall für $X$ mit Konfidenzniveau $1-\alpha$ ist
$$[\mu - z_{\alpha/2} \sigma, \mu + z_{\alpha/2} \sigma],$$
wobei $z_{\alpha/2}$ das $(1-\alpha/2)$-Quantil der Standardnormalverteilung ist.
Exponentiell transformiert ergibt sich das Konfidenzintervall für $Y$:
$$[e^{\mu - z_{\alpha/2} \sigma}, e^{\mu + z_{\alpha/2} \sigma}].$$

\begin{bsp}[Konfidenzintervall für den DAX]

Im folgenden R-Programm (Ausschnitt) wird ein 95\%-Konfidenzintervall für den DAX in einem Jahr von heute (252 Handelstage) berechnet.
Dazu werden die Parameter $\mu$ und $\sigma$ wie oben aus den täglichen log-Renditen geschätzt.

\begin{lstlisting}
alpha = 0.05
T <- 252
z <- qnorm(c(1 - alpha/2, alpha/2))
ci <- S0 * exp((mu - 0.5 * sigma^2) * T + z * sigma * sqrt(T))
\end{lstlisting}

Hier ist $S_0$ der heutige Kurswert des DAX. Das Konfidenzintervall lautet in diesem Fall:
$$[17217, 40097].$$

\end{bsp}

\begin{bsp}[Konfidenzband für den DAX]

Im folgenden R-Programm (Ausschnitt) wird ein 95\%-Konfidenzband für den DAX im naechsten Jahr (252 Handelstage) berechnet.

\begin{lstlisting}
alpha = 0.05
n <- 252
last_date <- max(dax$Date)
future_dates <- last_date + 1:n

q_low <- S0 * exp((mu - 0.5 * sigma^2) * (1:n) + qnorm(1 - alpha/2) * sigma * sqrt((1:n)))
q_hi  <- S0 * exp((mu - 0.5 * sigma^2) * (1:n) + qnorm(alpha/2) * sigma * sqrt((1:n)))
q_med <- S0 * exp((mu - 0.5 * sigma^2) * (1:n))

band <- data.frame(Date = future_dates, low = q_low, mid = q_med, hi = q_hi)
\end{lstlisting}

Es folgt eine Visualisierung des Konfidenzbandes im Anschluss an die historischen Daten.

\begin{figure}[H]
    \centering
    \includegraphics[width=0.9\textwidth]{images/dax_confidence_band.png}
    \caption{DAX mit 95\%-Konfidenzband für das nächste Jahr}
    \label{fig:dax_confidence_band}
\end{figure}

\end{bsp}

\subsection{Monte-Carlo-Simulation}

Genauso wie man eine Brownsche Bewegung mit summierten Normalverteilungen simuliert, 
wird eine geometrische Brownsche Bewegung durch exponentiell transformierte
summierte Normalverteilungen simuliert. 


\begin{bsp}[Monte-Carlo-Simulation des DAX]
Theoretisch liegt das stetige Modell zugrunde, 
aber in der Praxis wird eine diskrete Approximation verwendet, hier mit täglichen Schritten.
Der folgende R-Code (Ausschnitt) simuliert 1000 Pfade der geometrischen Brownschen Bewegung
mit den oben geschätzten Parametern $\mu$ und $\sigma$ für den DAX, wieder für das nächste Jahr (252 Handelstage).

\begin{lstlisting}
n <- 252
paths <- 10000
S0 <- tail(dax$Price, 1)

simulations <- replicate(paths, {
  W <- c(0, cumsum(rnorm(n, 0, 1)))
  S0 * exp((mu - 0.5 * sigma^2) *  c(0, (1:n)) + sigma * W)
})
\end{lstlisting}
Es folgt eine Visualisierung der Simulation im Anschluss an die historischen Daten.

\begin{figure}[H]
    \centering
    \includegraphics[width=0.8\textwidth]{images/dax_monte_carlo.png}
    \caption{DAX mit 10 simulierten Pfaden für das nächste Jahr}
    \label{fig:dax_monte_carlo}
\end{figure}

\end{bsp}

\begin{bsp}[Vergleich von Konfidenzintervall und Monte-Carlo-Simulation]

Man kann das Konfidenzintervall aus dem vorherigen Beispiel mit den quantilen der Monte-Carlo-Simulation vergleichen.

\begin{figure}[H]
    \centering
    \includegraphics[width=0.9\textwidth]{images/ci_comparison.png}
    \caption{Vergleich von Konfidenzintervall und Monte-Carlo-Simulation für verschiedene Simulationsanzahlen}
    \label{fig:ci_comparison}
\end{figure}
Man erkennt, dass das Konfidenzintervall mit steigender Simulationsanzahl immer besser durch die Quantile der Simulation approximiert wird.
Das spricht für die Korrektheit der beiden Verfahren.

\end{bsp}

\section{Aktienoptionen}

\subsection{Grundlagen aus der Wirtschaft}

Aktienoptionen sind Finanzderivate, die dem Inhaber das Recht, aber nicht die Pflicht geben, 
eine Aktie zu einem vorher festgelegten Preis (dem Ausübungspreis) zu kaufen (Call-Option) 
oder zu verkaufen (Put-Option). Europäische Optionen können nur am Fälligkeitstag ausgeübt 
werden, während amerikanische Optionen während der gesamten Laufzeit bis zum Verfallsdatum 
ausgeübt werden können. Ein zentrales Konzept ist die Put-Call-Parität, die eine Beziehung 
zwischen dem Preis einer europäischen Call- und Put-Option mit identischem Basiswert, 
Ausübungspreis und Laufzeit herstellt.

\begin{bsp}
\textit{Beispiel Call-Option.} 
Ein Investor erwirbt eine europäische Call-Option auf die Aktie der Firma X mit einem
 Ausübungspreis von 50\,€. Am Fälligkeitstag steht der Aktienkurs bei 60\,€. 
 Der Investor übt die Option aus, kauft die Aktie für 50\,€ und kann 
 sie sofort für 60\,€ verkaufen. Sein Gewinn (ohne Berücksichtigung der Optionsprämie) 
 beträgt 10\,€ pro Aktie.\\
\textit{Motivation.} Der Investor spekuliert darauf, dass der Kurs der Aktie steigt 
und über den Ausübungspreis hinausgeht.

\textit{Beispiel Put-Option.}
Ein Landwirt sichert sich gegen fallende Weizenpreise ab und kauft eine europäische 
Put-Option mit einem Ausübungspreis von 200\,€/Tonne. Am Fälligkeitstag liegt der 
Marktpreis bei 170\,€/Tonne. Der Landwirt übt die Option aus und verkauft seinen 
Weizen zum höheren Preis von 200\,€/Tonne, obwohl der Marktpreis niedriger ist. 
Sein Vorteil beträgt 30\,€ pro Tonne (abzüglich der Optionsprämie).
\end{bsp}


\section{Stochastische Analysis und alternative Kursmodelle}

\subsection{Stochastische Differentialgleichungen}

\subsection{Alternative Kursmodelle}

\subsection{Numerische Lösung stochastischer Differentialgleichungen}

\subsection{Vergleich der Modelle mit Backtests}

\section{Fazit}

Die Arbeit spannte den Bogen von elementaren stochastischen Prozessen über das Binomialmodell und die diskrete brownsche 
Bewegung bis zur geometrischen brownschen Bewegung als kontinuierlichem Grenzfall. Zentrale stochastische Begriffe wie 
Filtration, bedingter Erwartungswert und Martingal werden eingeführt und in diskreten Wahrscheinlichkeitsräumen verankert. 
Durch Grenzwertbetrachtungen werden die Konzepte dann auf kontinuierliche Wahrscheinlichkeitsräume übertragen. Es wird gezeigt, dass im Grenzübergang $\Delta t \to 0$ die Kursdynamik durch
$S_T = S_0 \exp\!\big((\mu - \tfrac12\sigma^2)T + \sigma W_T\big)$
beschrieben wird, sodass $\log S_T$ normal- und $S_T$ log-normalverteilt ist. 

Im erweiterten Binomialmodell wird der diskontierte Aktienkurs als Martingal unter einem risikoneutralen Maß konstruiert und die Optionsbewertung 
via Rückwärtsinduktion hergeleitet; im Grenzfall führt dies zum Black–Scholes-Modell.

Empirisch werden die Parameter aus 
Log-Renditen geschätzt, daraus Konfidenzintervalle und -bänder abgeleitet und mittels Monte-Carlo-Simulation validiert. 
Ein Backtest auf DAX-Daten zeigte eine hohe Überdeckungsrate im 50\%-Band und nachvollziehbare Fehlermaße (MSE, MAPE, NRMSE), 
was die Praxistauglichkeit trotz der Modellvereinfachungen unterstreicht. 

Letztlich wird gezeigt, wie die diskrete Modellierung durch Grenzübergang zur stochastischen Differentialgleichung führt, deren Lösung die 
geometrische brownsche Bewegung ist. Die Konstruktion des Itô-Integrals wird skizziert und die 
heuristische Schreibweise $dS_t = a(S_t,t)\,dt + b(S_t,t)\,dW_t$ mathematisch präzisiert.

Darauf aufbauend werden alternative Modelle wie lokale und stochastische Volatilität (z.B. CEV- und Heston-Modell), Sprung-Diffusions- und Regimewechselmodelle vorgestellt, die realistische Markteigenschaften wie Volatilitäts-Clustering oder Sprünge abbilden können. 
Am Beispiel des CEV-Modells wird die Parameterschätzung aus diskreten Daten mittels (Quasi-)Maximum-Likelihood 
erläutert und in R implementiert. Ein Vergleich von DAX, Lufthansa und dem Wechselkurs der türkischen Lira zeigt, dass GBM und CEV meist ähnliche Ergebnisse liefern, das CEV-Modell jedoch insbesondere bei ausgeprägter Korrelation zwischen Preis und Volatilität deutliche Vorteile bietet. Die Ergebnisse unterstreichen die Bedeutung der
 Modellauswahl und Kalibrierung für die praktische Anwendung in der Finanzmathematik.


\paragraph{Methodik}
Methodisch verbindet die Arbeit diskrete Grenzwert- und Martingalargumente mit 
reproduzierbarer Empirie in R. Zudem spielt die Monte-Carlo-Simulation eine zentrale Rolle als universelles 
Werkzeug zur Approximation risikoneutraler Erwartungswerte,
sowohl für Endwert-Auszahlungen als auch für pfadabhängige Optionen. Für die geometrische brownsche 
Bewegung werden Endwerte unter dem risikoneutralen Maß exakt berechnet; bei Pfadabhängigkeiten wird zeitlich diskretisiert.
Die beobachtete Annäherung der empirischen Quantile an die analytischen Lognormal-Quantile mit wachsender Pfadzahl bestätigt 
Konsistenz und Korrektheit der Implementierung; der Vergleich analytischer Konfidenzbänder mit Simulationsquantilen dient als robuster Plausibilitätscheck.
Als Beispiel dient ein Datensatz von DAX-Renditen.

Die geometrische brownsche Bewegung wird über Logarithmierung, Taylor-Entwicklung bis Ordnung zwei sowie Gesetz der großen Zahlen und 
zentralen Grenzwertsatz hergeleitet. Zudem kommen die Sätze von Cramér-Wold und Pratt zum Einsatz, 
genauso wie das Reihenkriterium für fast sichere Konvergenz, welches 
ein Korollar zum Lemma von Borel-Cantelli ist (vgl. \cite{henze}). Insbesondere wird das Kalkül der stochastischen Differentialgleichung zunächst vermieden.

Einen Einblick in die Thematik bietet Kapitel 8: die Methodik wird um die numerische Simulation und Parameterschätzung stochastischer Differentialgleichungen (SDEs) mittels Euler-Maruyama-Verfahren und Maximum-Likelihood-Ansätzen erweitert. 
Die Implementierung alternativer Modelle werden am Beispiel des CEV-Modell in R durchgeführt.
Dabei kamen die Pakete \texttt{Sim.DiffProc} und das numerische Optimierungspaket \texttt{nloptr} zum Einsatz.
Anschließend wird das CEV-Modell mit der GBM verglichen, inklusive Backtests mit empirischen Gütemaßen.


\section{Quellenverzeichnis}

\printbibliography

\end{document}