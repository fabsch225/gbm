\section{Notation}
\begin{itemize}
    \item $E(X)$ bzw. $E[X]$ - Erwartungswert
    \item $V(X)$ bzw. $V[X]$ - Varianz
    \item $X \sim N(\mu, \sigma^2)$ - normalverteilte Zufallsvariable mit Erwartungswert $\mu$ und Varianz $\sigma^2$
    \item $(\Omega, \mathcal F, P)$ - Wahrscheinlichkeitsraum
    \item $W_t$ - Brownsche Bewegung (Wiener-Prozess)
    \item GBM - geometrische Brownsche Bewegung
    \item CEV - Constant Elasticity of Variance (Modell)
    \item MSE - Mean Squared Error
    \item NRMSE - Normalized Root Mean Squared Error
    \item RMSE - Root Mean Squared Error
    \item MAPE - Mean Absolute Percentage Error
\end{itemize}

\paragraph{Konvergenzbegriffe}

\begin{itemize}
    \item $X_n \xrightarrow{pktw.} X$ - punktweise Konvergenz (Für jedes $\omega \in \Omega$ gilt: $X_n(\omega) \to X(\omega)$)
    \item $X_n \xrightarrow{glm.} X$ - gleichmäßige Konvergenz (Für jede $\epsilon > 0$ gilt: $\sup_{\omega \in \Omega} |X_n(\omega) - X(\omega)| < \epsilon$ für $n$ groß genug)
    \item $X_n \xrightarrow{d} X$ - Konvergenz in Verteilung (Die Verteilungsfunktion $F_{X_n}$ von $X_n$ konvergiert punktweise gegen die Verteilungsfunktion $F_X$ von $X$, mindestens an den Stetigkeitsstellen von $F_X$)
    \item $X_n \xrightarrow{p} X$ - Konvergenz in Wahrscheinlichkeit (Für jede $\epsilon > 0$ gilt: $P(|X_n - X| > \epsilon) \to 0$)
    \item $X_n \xrightarrow{f.s.} X$ - fast sichere Konvergenz (Für fast alle $\omega \in \Omega$ gilt: $X_n(\omega) \to X(\omega)$)
\end{itemize}

Auf das Verhältnis zwischen den Konvergenzbegriffen wird bei Bedarf eingegangen.