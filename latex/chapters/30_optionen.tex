\section{Aktienoptionen}

\subsection{Grundlagen aus der Wirtschaft}

Aktienoptionen sind Finanzderivate, die dem Inhaber das Recht, aber nicht die Pflicht geben, 
eine Aktie zu einem vorher festgelegten Preis (dem Ausübungspreis) zu kaufen (Call-Option) 
oder zu verkaufen (Put-Option). Europäische Optionen können nur am Fälligkeitstag ausgeübt 
werden, während amerikanische Optionen während der gesamten Laufzeit bis zum Verfallsdatum 
ausgeübt werden können. Ein zentrales Konzept ist die Put-Call-Parität, die eine Beziehung 
zwischen dem Preis einer europäischen Call- und Put-Option mit identischem Basiswert, 
Ausübungspreis und Laufzeit herstellt. Den fairen Preis einer Option zu bestimmen ist eine Herausforderung,
die man mit Hilfe der geometrischen Brownschen Bewegung lösen kann. Im Folgenden wird dazu das Black-Scholes-Modell 
vorgestellt.

\begin{bsp}
\textit{Beispiel Call-Option.} 
Ein Investor erwirbt eine europäische Call-Option auf die Aktie der Firma X mit einem
 Ausübungspreis von 50\,€. Am Fälligkeitstag steht der Aktienkurs bei 60\,€. 
 Der Investor übt die Option aus, kauft die Aktie für 50\,€ und kann 
 sie sofort für 60\,€ verkaufen. Sein Gewinn (ohne Berücksichtigung der Optionsprämie) 
 beträgt 10\,€ pro Aktie.\\
\textit{Motivation.} Der Investor spekuliert darauf, dass der Kurs der Aktie steigt 
und über den Ausübungspreis hinausgeht.

\textit{Beispiel Put-Option.}
Ein Landwirt sichert sich gegen fallende Weizenpreise ab und kauft eine europäische 
Put-Option mit einem Ausübungspreis von 200\,€/Tonne. Am Fälligkeitstag liegt der 
Marktpreis bei 170\,€/Tonne. Der Landwirt übt die Option aus und verkauft seinen 
Weizen zum höheren Preis von 200\,€/Tonne, obwohl der Marktpreis niedriger ist. 
Sein Vorteil beträgt 30\,€ pro Tonne (abzüglich der Optionsprämie).
\end{bsp}

\subsection{Bewertung von Aktienoptionen im Binomialmodell}

Das Binomialmodell wird erweitert. Zusätzlich zur Aktie und deren Kurs gibt es nun die Bank, die einen risikofreien Zinssatz $r$ anbietet.

\begin{defi}[Arbitrage-Prinzip]
Es wird angenommen, dass man keine risikolosen Gewinne erzielen kann, ohne Kapital zu investieren.
\end{defi}

\begin{defi}[Europäische Optionen]
    Europäische Optionen sind Finanzderivate, die dem Inhaber das Recht geben, 
    einen Basiswert zu einem festgelegten Preis (dem Ausübungspreis) nur am 
    Fälligkeitstag zu kaufen (Call-Option) oder zu verkaufen (Put-Option).
\end{defi}


\begin{lemma}[Risikoneutrale Wahrscheinlichkeit für einen Schritt]
Da in diesem Fall die Aussicht auf Zinsen berücksichtigt werden muss,
werden mögliche Aktiengewinne mit dem Faktor $e^{r \Delta t}$ verkleinert (diskontiert).
Der diskontierte Aktienkurs zum Zeitpunkt $n$ ist $e^{-r n \Delta t} S_n$.
Im Folgenden wird berechnet, für welche Wahrscheinlichkeit der neue diskontierte Prozess
ein Martingal ist. Das Martingal-Kriterium lautet: $E(X_{n+1} \mid X_n = l) = l$, wobei $X_n = e^{-r n \Delta t} S_n$.
Einsetzen von $X_{n+1} = e^{-r (n+1) \Delta t} S_{n+1}$ und $X_n = e^{-r n \Delta t} S_n$ ergibt
$$
E(e^{-r (n+1) \Delta t} S_{n+1} \mid S_n = v) = e^{-r n \Delta t} v
$$
Aus der Linearität des Erwartungswerts folgt
$$
E(e^{-r \Delta t} S_{n+1} \mid S_n=v) = v.
$$
Setzt man die möglichen Werte für $S_{n+1}$ ein ($S_{n+1} = u S_n$ mit Wahrscheinlichkeit $q$, $S_{n+1} = d S_n$ mit Wahrscheinlichkeit $1-q$) folgt
$$
e^{-r \Delta t} \left( q u S_n + (1-q) d S_n \right) = v.
$$
Teilt man durch $S_n = v$ ($v > 0$) ergibt sich
$$
e^{-r \Delta t} \left( q u + (1-q) d \right) = 1.
$$
Das ist äquivalent zu
$$
q = \frac{e^{r \Delta t} - d}{u - d}.
$$
Zum Vergleich: im klassischen Binomialmodell ist 
$$p = \frac{1 - d}{u - d}$$
die "risikoneutrale Wahrscheinlichkeit".
\end{lemma}

\begin{satz}[Bewertung von europäischen Optionen im Binomialmodell]
Der Preis einer Option durch rekursive Rückwärtsinduktion bestimmt. 
Der Wert einer europäischen Option am Endzeitpunkt $T$ ist durch die Auszahlungsfunktion $f(S_T)$ gegeben, z.\,B.\ für eine Call-Option $f(S_T) = \max(S_T - K, 0)$.
Die Bewertung erfolgt rekursiv rückwärts:
$$
C_n = e^{-r \Delta t} \left( q C_{n+1}^\text{up} + (1-q) C_{n+1}^\text{down} \right),
$$
wobei $C_{n+1}^\text{up}$ und $C_{n+1}^\text{down}$ die Optionswerte im nächsten Schritt nach Auf- bzw. Abbewegung sind.
Damit lässt sich der Optionspreis am Anfangszeitpunkt $C_0$ bestimmen.
\end{satz}

\subsection{Das Black-Scholes-Modell}
Ziel ist es zu zeigen, dass die Optionspreise im Cox–Ross–Rubinstein-(CRR-)Binomialmodell
beim Verfeinern der Zeitdiskretisierung gegen die Black–Scholes-Formel konvergieren.

\paragraph{CRR-Skalierung und risikoneutrales Maß}
Wähle für Zeitschritt $\Delta t$ die Sprunggrößen
$
u = e^{\sigma \sqrt{\Delta t}},\quad d = e^{-\sigma \sqrt{\Delta t}}
$
und den risikoneutralen Schritt $q(\Delta t)$ aus dem Martingal-Kriterium
$
q(\Delta t) = \frac{e^{r \Delta t} - d}{u - d}
= \frac{e^{r \Delta t} - e^{-\sigma \sqrt{\Delta t}}}{e^{\sigma \sqrt{\Delta t}} - e^{-\sigma \sqrt{\Delta t}}}.
$
Eine Taylor-Entwicklung liefert
$
q(\Delta t) = \tfrac12 + \frac{r - \tfrac12 \sigma^2}{2\sigma}\,\sqrt{\Delta t} + o(\sqrt{\Delta t}).
$

\paragraph{Grenzwert des Aktienprozesses}
Für $n=T/\Delta t$ und $N_u$ Aufwärtsbewegungen gilt unter $\mathbb Q$:
$
S_T^{(n)} = S_0\, u^{N_u}\, d^{n-N_u}
= S_0 \exp\!\big(\sigma \sqrt{\Delta t}\,(2N_u - n)\big).
$
Mit $N_u \sim \mathrm{Bin}(n, q(\Delta t))$ und dem ZGWS ergibt sich
$
2N_u - n
= (2q(\Delta t)-1)\,n + 2\sqrt{n\,q(\Delta t)(1-q(\Delta t))}\,Z_n,
$
wobei $Z_n \Rightarrow Z \sim \mathcal N(0,1)$. Multipliziert mit $\sigma \sqrt{\Delta t}$ folgt wegen $n\Delta t = T$:
\[
\log S_T^{(n)}
= \log S_0 + \big(r - \tfrac12 \sigma^2\big)T + \sigma \sqrt{T}\,Z_n + o(1)
\longrightarrow_{n\to\infty}
\log S_T \sim \mathcal N\!\Big(\log S_0 + \big(r - \tfrac12 \sigma^2\big)T,\;\sigma^2 T\Big).
\]
Damit konvergiert unter $\mathbb Q$ der diskrete Prozess gegen die Lösung der SDE
$
dS_t = r S_t\,dt + \sigma S_t\,dW_t
$
(geometrische Brownsche Bewegung).

\begin{satz}[Black–Scholes-Formel als Grenzwert des diskontierten Binomialmodells]
Sei $C_0^{(n)}$ der Preis der europäischen Call-Option mit Laufzeit $T$ und Strike $K$
im $n$-stufigen CRR-Modell unter Diskontierung mit $r$. Dann gilt
\[
\lim_{n\to\infty} C_0^{(n)}
= C_0^{\mathrm{BS}}
= S_0\,\Phi(d_1) - K e^{-rT}\,\Phi(d_2),
\]
wobei
\[
d_1 = \frac{\ln(S_0/K) + \big(r + \tfrac12 \sigma^2\big)T}{\sigma \sqrt{T}},
\qquad
d_2 = d_1 - \sigma \sqrt{T},
\]
und $\Phi$ die Verteilungsfunktion der Standardnormalverteilung ist.
\textit{Beweis}-
Risikoneutrale Bewertung im Binomialmodell liefert
$
C_0^{(n)} = e^{-rT}\,\mathbb E_{\mathbb Q}\!\left[(S_T^{(n)} - K)^+\right].
$
Die obige Verteilungskonvergenz impliziert
$
S_T^{(n)} \Rightarrow S_T = S_0 \exp\!\big((r-\tfrac12\sigma^2)T + \sigma \sqrt{T}\,Z\big).
$
Da $(x\mapsto (x-K)^+)$ beschränkt wachsend ist und die Familie $(S_T^{(n)})_n$ integrierbar bleibt,
darf Grenzübergang und Erwartungswert vertauscht werden (z.\,B.\ über gleichgradige Integrierbarkeit):
\[
\lim_{n\to\infty} C_0^{(n)} = e^{-rT}\,\mathbb E_{\mathbb Q}\!\left[(S_T - K)^+\right].
\]
Die rechte Seite ist die Lognormal-Erwartung und ergibt nach elementarer Rechnung die genannte Formel.
\end{satz}

\begin{bem}
Für den europäischen Put folgt analog
$
P_0^{\mathrm{BS}} = K e^{-rT}\,\Phi(-d_2) - S_0\,\Phi(-d_1),
$
oder per Put-Call-Parität. Zudem konvergieren die ganzen Preisbäume des CRR-Modells lokal gegen die Lösung der Black–Scholes-PDE.
\end{bem}