\section{Aktienoptionen}

\subsection{Grundlagen aus der Wirtschaft}

Aktienoptionen sind Finanzderivate, die dem Inhaber das Recht, aber nicht die Pflicht geben, 
eine Aktie zu einem vorher festgelegten Preis (dem Ausübungspreis) zu kaufen (Call-Option) 
oder zu verkaufen (Put-Option). Europäische Optionen können nur am Fälligkeitstag ausgeübt 
werden, während amerikanische Optionen während der gesamten Laufzeit bis zum Verfallsdatum 
ausgeübt werden können. Ein zentrales Konzept ist die Put-Call-Parität, die eine Beziehung 
zwischen dem Preis einer europäischen Call- und Put-Option mit identischem Basiswert, 
Ausübungspreis und Laufzeit herstellt.

\begin{bsp}
\textit{Beispiel Call-Option.} 
Ein Investor erwirbt eine europäische Call-Option auf die Aktie der Firma X mit einem
 Ausübungspreis von 50\,€. Am Fälligkeitstag steht der Aktienkurs bei 60\,€. 
 Der Investor übt die Option aus, kauft die Aktie für 50\,€ und kann 
 sie sofort für 60\,€ verkaufen. Sein Gewinn (ohne Berücksichtigung der Optionsprämie) 
 beträgt 10\,€ pro Aktie.\\
\textit{Motivation.} Der Investor spekuliert darauf, dass der Kurs der Aktie steigt 
und über den Ausübungspreis hinausgeht.

\textit{Beispiel Put-Option.}
Ein Landwirt sichert sich gegen fallende Weizenpreise ab und kauft eine europäische 
Put-Option mit einem Ausübungspreis von 200\,€/Tonne. Am Fälligkeitstag liegt der 
Marktpreis bei 170\,€/Tonne. Der Landwirt übt die Option aus und verkauft seinen 
Weizen zum höheren Preis von 200\,€/Tonne, obwohl der Marktpreis niedriger ist. 
Sein Vorteil beträgt 30\,€ pro Tonne (abzüglich der Optionsprämie).
\end{bsp}
