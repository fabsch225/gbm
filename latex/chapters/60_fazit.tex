\section{Fazit}

\subsection{Zusammenfassung}

Die Arbeit spannte den Bogen von elementaren stochastischen Prozessen über das Binomialmodell und die diskrete Brownsche 
Bewegung bis zur geometrischen Brownschen Bewegung als kontinuierlichem Grenzfall. Zentrale stochastische Begriffe wie 
Filtration, bedingter Erwartungswert und Martingal wurden eingeführt und in diskreten Wahrscheinlichkeitsräumen verankert. 
Durch Grenzwertbetrachtungen werden die Konzepte dann auf kontinuierliche Wahrscheinlichkeitsräume übertragen. Über eine 
Logarithmierung und eine systematische Taylor-Approximation der diskreten Renditen wurde gezeigt, dass im Grenzübergang 
$\Delta t \to 0$ die Kursdynamik durch
$S_T = S_0 \exp\!\big((\mu - \tfrac12\sigma^2)T + \sigma W_T\big)$
beschrieben wird, sodass $\log S_T$ normal- und $S_T$ log-normalverteilt ist. Im erweiterten Binomialmodell 
wurde der diskontierte Aktienkurs als Martingal unter einem risikoneutralen Maß konstruiert und die Optionsbewertung 
via Rückwärtsinduktion hergeleitet; im Grenzfall führt dies zum Black–Scholes-Modell. Empirisch wurden die Parameter aus 
Log-Renditen geschätzt, daraus Konfidenzintervalle und -bänder abgeleitet und mittels Monte-Carlo-Simulation validiert. 
Ein Backtest auf DAX-Daten zeigte eine hohe Überdeckungsrate im 50\%-Band und nachvollziehbare Fehlermaße (MSE, MAPE, NRMSE), 
was die Praxistauglichkeit trotz der Modellvereinfachungen unterstreicht.

\paragraph{Monte-Carlo-Simulation}
Monte-Carlo diente als universelles Werkzeug zur Approximation risikoneutraler
Erwartungswerte, sowohl für Endwert-Auszahlungen als auch für pfadabhängige Produkte.
Für die geometrische Brownsche Bewegung lassen sich Endwerte unter dem risikoneutralen Maß exakt ziehen; bei Pfadabhängigkeiten
wird zeitlich diskretisiert. Die beobachtete Annäherung der empirischen Quantile an die
analytischen Lognormal-Quantile mit wachsender Pfadzahl bestätigt Konsistenz und Korrektheit
der Implementierung. Der Vergleich analytischer Konfidenzbänder mit 
Simulationsquantilen ist ein einfacher, robuster Plausibilitätscheck.


\paragraph{Methodik}
Methodisch verbindet die Arbeit diskrete Grenzwert- und Martingalargumente mit 
reproduzierbarer Empirie in R. Als Beispiel dient ein Datensatz von DAX-Renditen.
Die geometrische Brownsche Bewegung wird über Logarithmierung, Taylor-Entwicklung bis Ordnung zwei sowie Gesetz der großen Zahlen und 
zentralen Grenzwertsatz hergeleitet. Zudem kommen die Sätze von Kolmogorov, und Pratt zum Einsatz, 
genauso wie das "Reihenkriterium für fast sichere Konvergenz", welches 
ein Korollar zum Lemma von Borel-Cantelli ist (vgl. \cite{henze}). Insbesondere wurde
das Kalkül der stochastischen Differentialgleichung vermieden. Einen Einblick in die Thematik bietet der
folgende Abschnitt.

\subsection{Weiterführende mathematische Aspekte}
Eine gründliche Auseinandersetzung derselben liefert \cite{shreve}.

\paragraph{Maßtheorie}
Eine strengere Ausarbeitung würde die risikoneutrale Bewertung explizit maßtheoretisch fassen: 
Arbitragefreiheit entspricht der Existenz eines äquivalenten Martingalmaßes, Bewertung erfolgt als 
Erwartungswert der diskontierten Auszahlung unter diesem Maß. Technisch geschieht der 
Maßwechsel über die Radon–Nikodym-Ableitung. Dieses Ergebnis wird auch Fundamentalsatz der Arbitragepreistheorie genannt. 
In kontinuierlicher Zeit vermittelt der Satz von Girsanov die Driftelimination. Diese Arbeit entschied sich zugunsten einer anschaulichen, aber 
konsistenten Herleitung.

\paragraph{Stochastische Analysis}
Stochastische Differentialgleichungen modellieren zeitstetige Zufallsdynamiken in der 
Form $dX_t = a(t,X_t)\,dt + b(t,X_t)\,dW_t$, wobei der stochastische Term über das It\^o-Integral interpretiert wird.
Das It\^o-Integral $\int_0^t \phi_s\,dW_s$ wird zunächst für elementare, an die natürliche Filtration adaptierte 
Treppenfunktionen $\phi_s = \sum_k \phi_k\,\mathbf{1}_{(t_k,t_{k+1}]}(s)$ definiert durch $\sum_k \phi_k\,(W_{t_{k+1}}-W_{t_k})$; 
mittels It\^o-Isometrie $E\!\left[\left(\int_0^t \phi_s\,dW_s\right)^2\right] = E\!\left[\int_0^t \phi_s^2\,ds\right]$ erfolgt dann die 
Fortsetzung auf quadratintegrable, vorhersagbare Prozesse als $L^2$-Abschluss. Lösungen von stochastischen Differentialgleichungen werden im Integralsinn 
verstanden, $X_t = X_0 + \int_0^t a(s,X_s)\,ds + \int_0^t b(s,X_s)\,dW_s$; 
unter Lipschitz- und Linearwachstumsbedingungen existieren eindeutige starke Lösungen. Eine zugängliche
Herleitung bietet \cite{behrends}.

Die geometrische Brownsche Bewegung ist die Lösung der stochastischen Differentialgleichung
$dS_t=\mu S_t\,dt+\sigma S_t\,dW_t$ mit $S_0=s_0$,
und die Black–Scholes-Bewertung lässt sich äquivalent über It\^o-Kalkül, Feynman–Kac und die Black–Scholes-PDE (partiellen Differentialgleichung)
$\partial_t V+\tfrac12\sigma^2 S^2 \partial_{SS}V + r S \partial_S V - r V=0$
begründen. Die vorliegende Darstellung ersetzte It\^o-Lemma und stochastische Integrale durch Taylor-Entwicklungen, Grenzwertbetrachtungen und 
Verteilungskonvergenz; beide Zugänge sind kompatibel, der Differentialgleichungs-Ansatz ist der etablierte Standard. 

\paragraph{Numerische Lösung stochastischer Differentialgleichungen}
Für die geometrische Brownsche Bewegung sind Endwerte unter dem risikoneutralen Maß direkt simulierbar. Allgemein werden stochastische Differentialgleichungen über 
zeitdiskrete Verfahren wie Euler–Maruyama (stark Ordnung 0{,}5, schwach 1) oder 
Milstein (stark Ordnung 1) approximiert. Für die Black–Scholes-PDE sind 
Finite-Differenzen-Verfahren (explizit, implizit, Crank–Nicolson) etabliert. Die Wahl zwischen 
Monte Carlo und partiellen Differentialgleichungen hängt von Dimension, Randbedingungen und Pfadabhängigkeit ab; in höheren 
Dimensionen und bei komplexen Auszahlungsprofilen ist Monte Carlo oft überlegen, während 
Differentialgleichungs-Ansätze für niedrigdimensionale, glatte Probleme sehr effizient sind.

\subsection{Ausblick}

Alternative Modelle adressieren beobachtete Abweichungen von der geometrischen Brownschen Bewegung wie variable Volatilität
und dünne Tails (Ränder). Stochastische Volatilität (z.\,B.\ Heston) bildet Volatilitäts-Cluster und 
Smile-Effekte ab, Sprung-Diffusion (Merton, Kou) modelliert diskontinuierliche Bewegungen und 
schwere Ränder, lokale Volatilität (Dupire) kalibriert Marktkonsistenz in der Volatilitätsfläche,
und Regimewechsel- oder GARCH-Modelle fangen Nichtstationaritäten in diskreter Zeit ein. 
Methodisch bieten Varianzreduktion und Quasi–Monte–Carlo Effizienzgewinne, Longstaff–Schwartz 
erleichtert die Bewertung amerikanischer Optionen, und Bootstrap-Verfahren 
quantifizieren Kalibrierungsunsicherheiten. Für die empirische Anwendung empfiehlt sich eine 
robuste, rollierende Kalibrierung, systematische Backtests über verschiedene Marktphasen und die 
Berücksichtigung mehrerer Gütemaße, um Prognosegüte und Modellrisiken transparent zu machen.