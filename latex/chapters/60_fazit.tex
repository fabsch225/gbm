\section{Fazit}

\subsection{Zusammenfassung}

Die Arbeit spannte den Bogen von elementaren stochastischen Prozessen über das Binomialmodell und die diskrete Brownsche 
Bewegung bis zur geometrischen Brownschen Bewegung (GBM) als kontinuierlichem Grenzfall. Zentrale stochastische Begriffe wie 
Filtration, bedingter Erwartungswert und Martingal wurden eingeführt und im diskreten Setting verankert. Über eine 
Logarithmierung und eine systematische Taylor-Approximation der diskreten Renditen wurde gezeigt, dass im Grenzübergang 
$\Delta t \to 0$ die Kursdynamik durch
$S_T = S_0 \exp\!\big((\mu - \tfrac12\sigma^2)T + \sigma W_T\big)$
beschrieben wird, sodass $\log S_T$ normal- und $S_T$ log-normalverteilt ist. Im erweiterten Binomialmodell 
wurde der diskontierte Aktienkurs als Martingal unter einem risikoneutralen Maß konstruiert und die Optionsbewertung 
via Rückwärtsinduktion hergeleitet; im Grenzfall führt dies zum Black–Scholes-Modell. Empirisch wurden die GBM-Parameter aus 
Log-Renditen geschätzt, daraus Konfidenzintervalle und -bänder abgeleitet und mittels Monte-Carlo-Simulation validiert. 
Ein Backtest auf DAX-Daten zeigte eine hohe Überdeckungsrate im 50\%-Band und nachvollziehbare Fehlermaße (MSE, MAPE, NRMSE) \cite{botchkarev_performance_2019}, 
was die Praxistauglichkeit trotz der Modellvereinfachungen unterstreicht.

\paragraph{Monte-Carlo-Simulation}
Monte-Carlo diente als universelles Werkzeug zur Approximation risikoneutraler 
Erwartungswerte, sowohl für Endwert-Auszahlungen als auch für pfadabhängige Produkte. 
Für GBM lassen sich Endwerte unter dem risikoneutralen Maß exakt ziehen; bei Pfadabhängigkeiten 
wird zeitlich diskretisiert. Die beobachtete Annäherung der empirischen Quantile an die 
analytischen Lognormal-Quantile mit wachsender Pfadzahl bestätigt Konsistenz und Korrektheit 
der Implementierung. In der Praxis bestimmen Schrittweitenwahl (Diskretisierungsfehler) und 
Varianzreduktion die Effizienz; der Vergleich analytischer Konfidenzbänder mit 
Simulationsquantilen ist ein einfacher, robuster Plausibilitätscheck.

\subsection{Weiterführende mathematische Aspekte}
Eine gründliche Auseinandersetzung liefert \cite{shreve}.

\paragraph{Maßtheorie}
Eine strengere Ausarbeitung würde die risikoneutrale Bewertung explizit maßtheoretisch fassen: 
Arbitragefreiheit entspricht der Existenz eines äquivalenten Martingalmaßes, Bewertung erfolgt als 
Erwartungswert der diskontierten Auszahlung unter diesem Maß. Technisch geschieht der 
Maßwechsel über die Radon–Nikodym-Ableitung. Dieses Ergebnis wird auch Fundamentalsatz der Arbitragepreistheorie genannt. 
In kontinuierlicher Zeit vermittelt der Satz von Girsanov die Driftelimination. Diese Arbeit entschied sich zugunsten einer anschaulichen, aber 
konsistenten Herleitung.

\paragraph{Stochastische Analysis}
Die GBM ist die Lösung der stochastischen Differentialgleichung
$dS_t=\mu S_t\,dt+\sigma S_t\,dW_t$ mit $S_0=s_0$,
und die Black–Scholes-Bewertung lässt sich äquivalent über It\^o-Kalkül, Feynman–Kac und die Black–Scholes-PDE
$\partial_t V+\tfrac12\sigma^2 S^2 \partial_{SS}V + r S \partial_S V - r V=0$
begründen. Die vorliegende Darstellung ersetzte It\^o-Lemma und stochastische Integrale durch Taylor-Entwicklungen, Grenzwertbetrachtungen und 
Verteilungskonvergenz; beide Zugänge sind kompatibel, der SDE/PDE-Ansatz ist der etablierte Standard. 
Zur Klarstellung: Unter allgemeinem Drift gilt $E(S_T)=S_0 e^{\mu T}$ und $\mathrm{Var}(S_T)=S_0^2 e^{2\mu T}\big(e^{\sigma^2 T}-1\big)$, 
und für Schrittweite $\Delta t$ sind die Log-Renditen normalverteilt mit Mittelwert $(\mu-\tfrac12\sigma^2)\Delta t$ und 
Varianz $\sigma^2 \Delta t$ (Standardabweichung $\sigma\sqrt{\Delta t}$).

\paragraph{Numerische Lösung stochastischer Differentialgleichungen}
Für GBM sind Endwerte unter dem risikoneutralen Maß direkt samplbar; allgemein werden SDEs über 
zeitdiskrete Verfahren wie Euler–Maruyama (stark Ordnung 0{,}5, schwach 1) oder 
Milstein (stark Ordnung 1) approximiert. Für die Black–Scholes-PDE sind 
Finite-Differenzen-Verfahren (explizit, implizit, Crank–Nicolson) etabliert. Die Wahl zwischen 
Monte Carlo und PDE hängt von Dimension, Randbedingungen und Pfadabhängigkeit ab; in höheren 
Dimensionen und bei komplexen Auszahlungsprofilen ist Monte Carlo oft überlegen, während 
PDE-Ansätze für niedrigdimensionale, glatte Probleme sehr effizient sind.

\subsection{Ausblick}

Alternative Modelle adressieren beobachtete Abweichungen der GBM-Annahmen wie konstante Volatilität
und dünne Tails. Stochastische Volatilität (z.\,B.\ Heston) bildet Volatilitäts-Cluster und 
Smile-Effekte ab, Sprung-Diffusion (Merton, Kou) modelliert diskontinuierliche Bewegungen und 
schwere Schwänze, lokale Volatilität (Dupire) kalibriert Marktkonsistenz in der Volatilitätsfläche,
und Regimewechsel- oder GARCH-Modelle fangen Nichtstationaritäten in diskreter Zeit ein. 
Methodisch bieten Varianzreduktion und Quasi–Monte–Carlo Effizienzgewinne, Longstaff–Schwartz 
erleichtert die Bewertung amerikanischer Optionen, und Bootstrap-Verfahren 
quantifizieren Kalibrierungsunsicherheit. Für die empirische Anwendung empfiehlt sich eine 
robuste, rollierende Kalibrierung, systematische Backtests über verschiedene Marktphasen und die 
Berücksichtigung mehrerer Gütemaße (\cite{Botchkarev}), um Prognosegüte und Modellrisiken transparent zu machen.